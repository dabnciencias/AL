\chapter*{Presentación del curso} \label{Chap: Presentación}

La presente planeación del curso de Álgebra Lineal para la Licenciatura en Física Biomédica fue realizada tomando en cuenta el existente programa de asignatura\footnote{El temario oficial puede ser consultado en la página \url{http://www.fciencias.unam.mx/asignaturas/1330.pdf}.}, así como la retroalimentación del mismo por parte de estudiantes de la Licenciatura.

\section*{Objetivo} \label{Ssec: Objetivo}

Durante el curso, se sentarán bases teóricas sólidas de álgebra lineal para diversas aplicaciones que serán útiles en materias del plan de estudios, tales como Cálculo Avanzado, Ecuaciones Diferenciales, Matemáticas Avanzadas, Introducción a la Física Cuántica y Mecánica Cuántica, Algoritmos Computacionales y Física Computacional, entre otras.

\section*{Metolodogía de enseñanza}

 Este curso está diseñado para que lxs estudiantes aprendan álgebra lineal de manera mayormente autodidacta, a la vez que se les da acompañamiento en los temas del curso tanto dentro como fuera del salón de clase. Desde el principio del semestre, lxs estudiantes tendrán acceso a las notas del curso, que incluyen ejercicios de repaso para reforzar los aprendizajes: esto permitirá que cada quien avance a su ritmo, y que puedan estudiar los temas que veremos en clase en caso de no asistir. Habrán cuatro clases y una ayudantía semanales en las cuales se presentarán los temas y se fomentará la discusión en grupo, así como horarios fijos establecidos fuera de clase para atender dudas de la materia de forma más personalizada (posiblemente, de manera virtual).

 \vspace{3mm}
  Como plataforma para llevar el curso, utilizaremos la aplicación de mensajería instantánea Telegram -misma por donde publicaremos avisos, recursos didácticos (especificados más abajo), exámenes, etc. y atenderemos dudas de nuestrxs estudiantes fuera del horario de clases.

\vspace{3mm}
Durante todo el curso, se procurará la interacción y colaboración entre estudiantes durante el proceso de aprendizaje.

\section*{Temario del curso por semana}

\subsection*{Módulo 1: Nociones básicas del álgebra lineal} \label{Ssec: Módulo 1: Nociones básicas del álgebra lineal}

\textbf{Semana 1:}  Estructuras algebraicas, campos, espacios vectoriales y subespacios vectoriales.

\vspace{3mm}
\textbf{Semana 2:}  Combinaciones lineales, conjunto generador y subespacio generado.

\vspace{3mm}
\textbf{Semana 3:}  Dependencia e independencia lineal, bases y dimensión.

\vspace{3mm}
\textbf{Semana 4:}  Transformaciones lineales.

\subsection*{Módulo 2: Representaciones de espacios vectoriales de dimensión finita} \label{Ssec: Representaciones de espacios vectoriales de dimensión finita}

\textbf{Semana 5:}  Representación de vectores como $n$-tuplas y transformaciones lineales como matrices. (bases ordenadas)

\vspace{3mm}
\textbf{Semana 6:} Representaciones vistas como isomorfismos. (operaciones entre transformaciones lineales y matrices)

\vspace{3mm}
\textbf{Semana 7:} Invertibilidad de matrices e isomorifsmos asociados.

\vspace{3mm}
\textbf{Semana 8:} Cambios de base.

\subsection*{Módulo 3: Caracterización algebraica de transformaciones lineales} \label{Ssec: Módulo 3: Caracterización algebraica de transformaciones linelaes}

\textbf{Semana 9:} Eigenvectores, eigenvalores y diagonalización.

\vspace{3mm}
\textbf{Semana 10:} Diagonalizabilidad, eigenespacios y descomposición espectral.

\subsection*{Módulo 4: Caracterización geométrica de operadores lineales} \label{Ssec: Módulo 4: Caracterización geométrica de operadores lineales}

\textbf{Semana 11:} Producto escalar, bases ortogonales y proyecciones vectoriales.

\vspace{3mm}
\textbf{Semana 12:} Bases ortonormales y norma inducida, ortogonalización y ortonormalización.

\vspace{3mm}
\textbf{Semana 13:} Funcionales y espacio dual, complemento ortogonal y proyecciones ortogonales.

\vspace{3mm}
\textbf{Semana 14:} Operadores adjuntos, normales y autoadjuntos.

\vspace{3mm}
\textbf{Semana 15:} Teorema espectral, operadores unitarios y ortogonales.

\vspace{3mm}
\textbf{Nota:} Debido al poco tiempo del que disponemos para el curso, \textbf{se asumirá que lxs estudiantes dominan los siguientes temas del} \href{https://web.fciencias.unam.mx/asignaturas/1130.pdf}{temario de Álgebra}: Matrices (definción y operaciones), matrices transpuestas, operaciones elementales y matrices elementales, rango de una matriz, matrices invertibles, cálculo de la inversa de una matriz invertible, determinante de una matriz cuadrada (definición y propiedades), cálculo de determinantes, soluciones de un sistema de ecuaciones lineales, sistemas de ecuaciones lineales homogéneos, espacio de soluciones de un sistema homogéneo, sistemas de ecuaciones lineales no homogéneos, criterios de existencia de soluciones y resolución de sistemas de ecuaciones lineales. %Sin embargo, \textbf{no se asumirá conocimiento alguno del temario de Ecuaciones Diferenciales}.

\section*{Forma de evaluación}

tareas-examen (4) 80\%

proyecto final  20\%

tareas morales     5\% extra

%La evaluación se hará a través de 4 tareas-exámenes. Las tareas-examen se realizarán y entregarán en equipos de pocas personas, que serán asignados por el profesor y cambiarán después de cada entrega; esto fomentará la amplia colaboración entre estudiantes, y permitirá que se le pueda brindar una atención más focalizada a cada equipo. Para asegurar que el trabajo sea colaborativo, en cada entrega se deberá indicar de forma clara las contribuciones que realizó cada miembro del equipo al trabajo entregado. La calificación que obtenga un equipo en una tarea-examen será la misma que se asigne a cada contribuyente para esa misma entrega. Quienes no contribuyan a su equipo en una entrega no tendrán derecho a la calificación correspondiente.
%
%Además de las cuatro tareas-examen, se evaluará un trabajo final obligatorio (que puede realizarse de forma individual o en parejas) en el cual realicen una investigación sobre aplicaciones de los temas vistos durante el curso y, en su caso, realicen una implementación numérica (i.e., un programa) acerca de la aplicación que investigaron. Este trabajo final se promediará con las cuatro tareas-examen para obtener la calificación final del curso.
%
%Al final del curso, quienes tengan un promedio reprobatorio en la materia deberán presentar un examen final de forma individual, que sustituirá sus calificaciones anteriores. Las personas con promedio aprobatorio también podrán optar por hacer el examen final para subir calificación; en ese caso, su calificación en el final reemplazará las anteriores.

\section*{Bibliografía recomendada para el curso} \label{Bibliografía}

\begin{itemize}
    \item S. H. Friedberg, \emph{Linear Algebra}, 4ta ed. (Pearson, 2014, EUA) - es el texto básico para este tipo de cursos.
    \item S. Lang, \emph{Linear Algebra}, 3a ed. (Springer, 1987, EUA) - buen complemento al Friedberg.
    \item D. Poole, \emph{Linear Algebra: A Modern Introduction}, 4ta ed. (Cengage Learning, 2015, EUA) - útil para quienes quieran ver algunas aplicaciones de los conceptos al mismo tiempo que los aprenden.
\end{itemize}{}

Les sugiero que hojeen \emph{todos} los libros recomendados al inicio del curso, y que consulten los de su agrado constantemente durante el mismo, o bien, busquen otros que les sirvan mejor para aprender.

\section*{Otros recursos educativos}

No sólo se aprende de libros; hay que aprovechar todo el contenido que ofrece el internet para nuestra educación. A lo largo de los apuntes pondré hipervínculos a algunas páginas con recursos relevantes para el tema en cuestión; sin embargo, aquí enlistaré algunos recursos útiles para aprender álgebra lineal:

\begin{itemize}
    \item Lista de reproducción \href{https://www.youtube.com/watch?v=fNk_zzaMoSs&list=PLZHQObOWTQDPD3MizzM2xVFitgF8hE_ab}{Essence of Linear Algebra} del canal de YouTube 3Blue1Brown.
    \item Lista de reproducción \href{https://youtube.com/playlist?list=PL91agCMqt_mdAgHZkxyn-tscoNpu7ZHvl}{Espacios vectoriales con producto escalar} del canal de YouTube Animathica.
    \item Libros de texto interactivos \href{{http://immersivemath.com/ila/learnmore.html}}{Immersive Linear Algebra} e \href{https://textbooks.math.gatech.edu/ila/index2.html}{Interactive Linear Algebra}, que sirven para generar intuición acerca de algunos conceptos del álgebra lineal.
    \item Series de videos de sobre álgebra lineal de \href{https://www.lem.ma/books/AIApowDnjlDDQrp-uOZVow/landing}{lemma} y \href{https://www.khanacademy.org/math/linear-algebra}{Khan Academy} con interfaces para resolver ejercicios al final de cada lección.
\end{itemize}

Por supuesto, les invito a que busquen más recursos por su propia cuenta; de encontrarlos útiles, les agradecería que me notificaran para revisarlos.
