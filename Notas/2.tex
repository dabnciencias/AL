\documentclass[notasLineal]{subfiles}
\begin{document}

\section{Subespacios vectoriales, combinaciones lineales, conjunto generador y subespacio generado} \label{Sec: Subespacios vectoriales, combinaciones lineales, conjunto generador y subespacio generado}

Anteriormente, vimos que es posible tomar a un subconjunto de un campo de tal manera que las operaciones del campo, restringidas al subconjunto, formen un campo, y llamamos a esto un \emph{subcampo}. Resulta que podemos hacer algo similar con espacios vectoriales, definiendo el concepto de \emph{subespacio vectorial}. Más aún, podemos dar condiciones explícitas para que un subconjunto del conjunto vectorial de un espacio forme un subespacio vectorial de dicho espacio.

\subsubsection*{Definición de subespacio vectorial} \label{Def: Subespacio vectorial}

\begin{tcolorbox}
    \underline{Def.} Sea $(V,K)$ un espacio vectorial. Si $S$ es un subcampo de $K$ y $W\subseteq V$ es tal que las operaciones de suma vectorial y producto de un vector por un escalar en $(V,K)$ restringidas a $W\times W$ y $S\times W$, respectivamente, forman un espacio vectorial, entonces decimos que $(W,S)$ es un \emph{subespacio vectorial} de $(V,K)$.
\end{tcolorbox}

\begin{Obs}\label{obs:1.8}
%    Si $S$ es un subcampo de $K$, entonces $(S,S)$ y $(K,S)$ son subespacios vectoriales de $(K,K)$. En efecto, esto se sigue de los Ejercicios \ref{ejer-4} y \ref{ejer-5}. En particular, de los ejemplos de esos ejercicios se sigue que $(\mathbb{R},\mathbb{R})$ y $(\mathbb{C},\mathbb{R})$ son subespacios vectoriales de $(\mathbb{C},\mathbb{C})$.

    Si $S$ es un subcampo de $K$, entonces $(S,S)$ es un subespacio vectorial de $(K,K)$. En efecto: Esto se sigue del Ejercicio \ref{ejer-4}. En particular, de los ejemplos de ese ejercicio se sigue que $(\mathbb{R},\mathbb{R})$ es un subespacio vectorial de $(\mathbb{C},\mathbb{C})$.
\end{Obs}

Si $(V,K)$ es un espacio vectorial y $W\subseteq V$, podemos preguntarnos: ¿Qué condiciones necesita cumplir el subconjunto $W$ para que forme un subespacio vectorial sobre $K$ de $(V,K)$? La respuesta está dada por el siguiente resultado, que nos da una caracterización muy útil de los subespacios vectoriales.

\begin{Prop}[Caracterización de subespacios vectoriales]\label{prop: Caracterización de subespacios vectoriales}
    
    Sean $(V,K)$ un espacio vectorial y $W\subseteq V$. Entonces las condiciones siguientes equivalen a que $(W,K)$ sea un subespacio vectorial de $(V,K)$:

    \begin{enumerate}[label=(\alph*)]

        \item $W$ es cerrado bajo la suma vectorial,

        \item $W$ es cerrado bajo el producto de un vector por un escalar, y

        \item $W$ contiene al vector nulo de $V$.
    \end{enumerate}
\end{Prop}

\begin{proof}\leavevmode
    
    Supongamos que $(W,K)$ es un subespacio vectorial de $(V,K)$. Entonces, $(W,K)$ forma un espacio vectorial con las operaciones de suma vectorial y producto de un vector por un escalar en $(V,K)$ restringidas a $W\times W$ y $K\times W$. Por definición de espacio vectorial, $W$ es cerrado bajo la suma vectorial y el producto de un vector por un escalar en $K$, por lo que se cumplen (a) y (b), y $W$ contiene a un vector nulo. Como $W\subseteq V$ y $V$, por definición de espacio vectorial, contiene a un vector nulo, entonces del inciso (c) de la Proposición \ref{prop:1.2}, se sigue que el vector nulo de $W$ es el mismo de $V$. \\

    Por otro lado, supongamos que $W$ cumple (a), (b) y (c). Puesto que $(V,K)$ es un espacio vectorial y $W\subseteq V$, entonces de (a) y (b) se sigue directamente que la suma vectorial en $V$ restringida a $W$ es asociativa y conmutativa, el producto de un vector por un escalar restringido a $K\times W$ es compatible con el producto entre escalares, existe un elemento identidad del producto de un vector por un escalar $1\in K$, y que el producto de un vector por un escalar restringido a $W$ y $K$ se distribuye con respecto a la suma vectorial y con respecto a la suma escalar. En particular, por la existencia de inversos aditivos en el campo $K$, la existencia del elemento $1\in K$ y la distributividad del producto de un vector por un escalar con respecto a la suma escalar, se sigue que existen inversos aditivos para todos los elementos de $W$. Finalmente, por (c), existe un neutro aditivo en $W$. Por lo tanto, $W$ es tal que las operaciones de suma vectorial y producto de un vector por un escalar restringidas a $W\times W$ y $K\times W$ forman un espacio vectorial, de donde se sigue que $(W,K)$ es un subespacio vectorial de $(V,K)$.
\end{proof}

\begin{Obs} \label{obs: Simplificación de la caracterización de subespacios vectoriales}
    Las condiciones (a) y (b) de la Proposición \ref{prop: Caracterización de subespacios vectoriales} son equivalentes a decir que, si $\mathbf{u},\mathbf{v}\in W$ y $a\in K$, entonces $\mathbf{u}+a\mathbf{v}\in W$.
\end{Obs}

\begin{Coro}\label{coro:2.3}
    Sean $(V,K)$ un espacio vectorial, $W\subseteq V$ y $S$ es un subcampo de $K$. Entonces, las condiciones de la Proposición \ref{prop: Caracterización de subespacios vectoriales} \textemdash restringiendo el producto de un vector por un escalar a $S\times W$\textemdash \ son equivalentes a que $(W,S)$ sea un subespacio vectorial de $(V,K)$.
\end{Coro}

\begin{proof}
    Es totalmente análoga a la de demostración de la Proposición \ref{prop: Caracterización de subespacios vectoriales}.
\end{proof}

\begin{Obs}\label{obs:2.4}\leavevmode
    \begin{enumerate}[label=(\arabic*)]

        \item Como todo subespacio vectorial es, en particular, un espacio vectorial, entonces cualquier subespacio vectorial puede tener subespacios vectoriales subsecuentes, todos con el mismo vector nulo.

        \item Para todo espacio vectorial $V$, $V$ y $\{\mathbf{0}\}$ son trivialmente subespacios vectoriales de $V$.
    \end{enumerate}
   
\end{Obs}

\subsubsection*{Ejemplos de subespacios vectoriales}

El conjunto de todos los pares ordenados $\{(x_1,x_2)\in\mathbb{R}^2 | x_1=x_2\}$ es un subespacio vectorial del espacio vectorial real $\mathbb{R}^2$. \\

Si $j,k\in\mathbb{N}$ son tales que $j<k$, entonces el conjunto de polinomios de grado $j$ es un subespacio vectorial\footnote{De aquí en adelante, asumiremos que cualquier espacio vectorial $V$ está definido por un conjunto vectorial $V$ sobre el campo real a menos que se indique lo contrario.} del espacio vectorial real de polinomios de grado $k$. \\

Si $n\in\mathbb{N}$, entonces el conjunto de todas las funciones reales de clase $C^{\infty}$ es un subespacio vectorial de $(C^n,\mathbb{R})$.

\subsection*{Intersección y suma de subespacios vectoriales}

A continuación, presentamos algunas operaciones que podemos realizar entre subespacios vectoriales de un cierto espacio vectorial para obtener nuevos subespacios de dicho espacio. En realidad, estas operaciones se realizarán entre los \emph{conjuntos vectoriales} de dichos subespacios, resultando en un conjunto de vectores que forma un subespacio vectorial sobre el mismo campo que el espacio vectorial original.

\begin{Teo}\label{teo: Intersección de dos subespacios vectoriales} Sea $V$ un espacio vectorial. Entonces, cualquier intersección de dos subespacios vectoriales de $V$ es un subespacio vectorial de $V$.

\begin{proof}
    Sea $V$ sobre $K$ un espacio vectorial y sean $W_1,W_2$ subespacios vectoriales de $V$. Por la Proposición \ref{prop: Caracterización de subespacios vectoriales}, cada subespacio vectorial de $V$ contiene al neutro aditivo de $V$, por lo que $\mathbf{0}\in W_1\cap W_2$. \\ 

    Sean $\mathbf{u},\mathbf{v}\in W_1\cap W_2$ y $a\in K$. Por la Observación \ref{obs: Simplificación de la caracterización de subespacios vectoriales}, cada subespacio contiene a $\mathbf{u}+a\mathbf{v}$, por lo que $\mathbf{u}+a\mathbf{v}\in W_1\cap W_2$. De lo anterior, concluimos que $W_1\cap W_2$ es un subespacio vectorial de $V$.
\end{proof}
\end{Teo}

\begin{Coro}\label{coro: Intersección finita de subespacios vectoriales}
    Cualquier intersección finita de subespacios vectoriales de un espacio vectorial $V$ es un subespacio vectorial de $V$.
\end{Coro}

\begin{proof}
    Se sigue del Teorema \ref{teo: Intersección de dos subespacios vectoriales} y de que la intersección de conjuntos es una operación asociativa.
\end{proof}

\begin{Ejer}\label{ejer: La relación de ser subespacio vectorial es transitiva}
    Sea $Z$ un subespacio vectorial de $W$ y sea $W$, a su vez, subespacio vectorial de $V$. Demuestra que $Z$ es un subespacio vectorial de $V$.
\end{Ejer}

\begin{tcolorbox}
    \underline{Def.} Sean $S_1$ y $S_2$ subespacios de un espacio vectorial $V$. Definimos a la \emph{suma de los subespacios vectoriales} $S_1$ y $S_2$ como
    \[
    S_1+S_2=\{\mathbf{x}+\mathbf{y}\mathop|\mathop \mathbf{x}\in S_1, \mathbf{y}\in S_2\}.
    \] 
\end{tcolorbox}

\begin{Ejer}\label{ejer: La suma de subespacios vectoriales es un subespacio vectorial}
    Demuestra que cualquier suma finita de subespacios vectoriales de un espacio vectorial $V$ es un subespacio vectorial de $V$.
\end{Ejer}

\subsubsection*{Combinaciones lineales, conjunto generador y subespacio generado} \label{Sssec: Combinaciones lineales, conjunto generador y subespacio generado}

Ahora, veremos más formas de obtener subespacios vectoriales a partir de un espacio vectorial. Sabemos que las operaciones necesarias para definir a un espacio vectorial son la suma vectorial y el producto de un vector por un escalar. La operación más general que podemos realizar a partir de dichas operaciones se define a continuación.

\begin{tcolorbox}
    \underline{Def.} Sea $(V,K)$ un espacio vectorial y $L=\{\mathbf{v}_1, \mathbf{v}_2, ..., \mathbf{v}_n\}\subseteq V$ un conjunto finito de vectores de $V$. Decimos que $\mathbf{u}$ es una \emph{combinación lineal} de los vectores de $L$ si existen escalares $c_i\in K$ para $i\in\{1,2,...,n\}$ tales que
    \[
    \mathbf{u} =c_1\mathbf{v}_1+c_2\mathbf{v}_2+...+c_n\mathbf{v}_n=\sum_{i=1}^n c_i\mathbf{v}_i.
    \] 
    En este caso, decimos que los escalares $c_i$ son los \emph{coeficientes} de la combinación lineal $\sum_{i=1}^n c_i\mathbf{v}_i$.
\end{tcolorbox}

\begin{Obs}\label{obs: Combinaciones lineales}\leavevmode
    \begin{enumerate}[label=(\arabic*)]
    
        \item Dadas las propiedades de cerradura de las operaciones esenciales de los espacios vectoriales, cualquier combinación lineal de vectores de un espacio vectorial $V$ resultará en un vector de $V$.

        \item El vector nulo de un espacio vectorial puede ser obtenido como combinación lineal de cualquier conjunto de vectores. Por el inciso (a) del Teorema \ref{teo:1.8}, basta fijar a todos los coeficientes de la combinación lineal como el neutro aditivo del campo. A este tipo de combinación lineal se le conoce como \emph{combinación lineal trivial}. 

        \item Siguiendo las interpretaciones geométricas de las operaciones de suma entre vectores y producto de un vector por un escalar vistas anteriormente, podemos interpretar a esta operación generalizada como la combinación de flechas (o líneas) reescaladas y posiblemente rotadas, si el espacio vectorial es complejo y el escalar tiene una parte imaginaria no nula, a las cuales aplicamos la Ley del paralelogramo para obtener una nueva flecha (o línea) como resultado. Precisamente por esta razón es que a esta operación general se le conoce como \emph{combinación lineal}.
    \end{enumerate}
\end{Obs}

\begin{Ejer}\label{ejer-9}
    Sea $(V,K)$ un espacio vectorial y $L=\{\mathbf{v}_1,\mathbf{v}_2,...,\mathbf{v}_n\}\subseteq V$ un conjunto finito de vectores de $V$. Demuestra que el conjunto de todas las combinaciones lineales de $L$
    \[
    \{c_1\mathbf{v}_1+c_2\mathbf{v}_2+...+c_n\mathbf{v}_n \mid c_i\in K\}
    \] 
    es un subespacio vectorial de $V$.
\end{Ejer}

En vista del Ejercicio \ref{ejer-9}, damos la siguiente definición.

\begin{tcolorbox} \label{Def: Conjunto generador y subespacio generado}
    \underline{Def.} Sea $V$ sobre $K$ un espacio vectorial y $L\subseteq V$ finito. Entonces, definimos al \emph{subespacio generado por} $L$ como
    \[
    \langle L\rangle := \{c_1\mathbf{v}_1+c_2\mathbf{v}_2+...+c_n\mathbf{v}_n\hspace{0.5mm}|\hspace{0.5mm}c_i\in K, \mathbf{v}_i\in L\}.
    \] 
    A $L$ se le conoce como el \emph{conjunto generador}. Por completez, definimos $\langle \emptyset \rangle = \{\mathbf{0}\}$.
\end{tcolorbox}

\begin{Obs}\label{obs: Notación de subespacios generados}
    A menudo denotaremos al subespacio generado por un conjunto de vectores $\{\mathbf{v}_1,\mathbf{v}_2,...,\mathbf{v}_n\}$ simplemente como $\langle\mathbf{v}_1,\mathbf{v}_2,...,\mathbf{v}_n\rangle$ en vez de $\langle\{\mathbf{v}_1,\mathbf{v}_2,...,\mathbf{v}_n\}\rangle$, cuando esto no lleve a una confusión.
\end{Obs}

En un espacio vectorial arbitrario es posible expresar a cualquiera de sus vectores como combinación lineal de otros vectores del mismo espacio. Por ejemplo, en $\mathbb{R}^2$ el vector $$\begin{pmatrix} 1 & 5 \end{pmatrix} = \begin{pmatrix} 1 & 0 \end{pmatrix} + 5\begin{pmatrix} 0 & 1 \end{pmatrix} = 2\begin{pmatrix} 1 & 1.5 \end{pmatrix} + (-0.5)\begin{pmatrix} 2 & -4 \end{pmatrix} $$ $$ = (-4)\begin{pmatrix} 0.5 & -3 \end{pmatrix} + 3\begin{pmatrix} 1 & 1 \end{pmatrix} + (-5)\begin{pmatrix} 0 & 2 \end{pmatrix} = ...$$ \noindent Observamos que, en cada caso, el valor de los coeficientes $c_i\in\mathbb{R}$ depende de los vectores $\mathbf{v}_i\in\mathbb{R}^2$ con los cuales se realiza la combinación lineal. Para dar otro ejemplo, en $P^2$, si definimos los vectores $f(x) = 7x^2 - 5x + 2, g(x) = x^2, h(x) = 9x, i(x)=7, j(x)=x^2 + x + 1$, podemos verificar que $$f(x) = 7g(x)-\frac{5}{9}h(x)+\frac{2}{7}i(x)=7j(x)-\frac{4}{3}h(x)+\frac{1}{7}i(x)=3j(x)+4g(x)+-\frac{8}{9}h(x)-\frac{1}{7}i(x)=...$$

Para dar algunos ejemplos, si elegimos cualquier vector $\mathbf{v}\in\mathbb{R}^2$, entonces el subespacio generado correspondiente $\langle \mathbf{v} \rangle = \{c\mathbf{v}\hspace{0.5mm}|\hspace{0.5mm}c\in \mathbb{R}\}$ se puede interpretar geométricamente en el plano cartesiano como el conjunto de todas las flechas posibles de obtener a partir de reescalamientos de $\mathbf{v}$. Por otro lado, si en $\mathbb{R}^3$ definimos a $N=\{\begin{pmatrix} 1 & 0 & 0 \end{pmatrix}, \begin{pmatrix} 0 & 1 & 0 \end{pmatrix}, \begin{pmatrix} 0 & 0 & 1 \end{pmatrix}\}$ entonces vemos que $$\langle N \rangle = \{c_1\begin{pmatrix} 1 & 0 & 0 \end{pmatrix} + c_2\begin{pmatrix} 0 & 1 & 0 \end{pmatrix} + c_3\begin{pmatrix} 0 & 0 & 1 \end{pmatrix}\hspace{0.5mm}|\hspace{0.5mm}c_1,c_2,c_3\in\mathbb{R}\},$$ pero esto es equivalente a la definición $\mathbb{R}^3 = \{\begin{pmatrix} c_1 & c_2 & c_3\end{pmatrix} \hspace{0.5mm}|\hspace{0.5mm}c_1,c_2,c_3\in\mathbb{R}\}$; es decir, en este caso \emph{el espacio generado por los vectores de $N$ es igual a $\mathbb{R}^3$}, i.e., $\langle N \rangle =\mathbb{R}^3$.

A continuación, veremos un teorema que será de gran importancia en las secciones posteriores.

\begin{Teo}\label{teo: Agregar un elemento de un subespcio generado al conjunto generador deja al subespacio generado invariante}
    Sea $V$ un espacio vectorial, $S\subseteq V$ un conjunto finito de vectores de $V$ y $\mathbf{v}\in V$ un vector arbitrario. Si $S'=S\cup\{\mathbf{v}\}$, entonces $\langle S \rangle = \langle S' \rangle \iff \mathbf{v}\in\langle S \rangle.$

\begin{proof}
Ya que $\mathbf{v}\in S'$ entonces trivialmente se cumple que $\mathbf{v}\in\langle S'\rangle;$ por lo tanto, si $\mathbf{v}\notin \langle S \rangle \implies \langle S \rangle \neq \langle S' \rangle.$ Por otro lado, si $\mathbf{v}\in\langle S \rangle$ entonces $S'\subset\langle S \rangle$, lo cual implica que $\langle S' \rangle \subset \langle S \rangle.$ Además, ya que $S\subset S'$, entonces trivialmente se cumple que $\langle S \rangle \subset \langle S' \rangle.$ En conclusión, $\langle S \rangle =\langle S' \rangle.$
\end{proof}
        
    Este teorema nos dice que agregar un vector a un conjunto generador no necesariamente cambiará el subespacio generado por ese conjunto generador. Para que este cambio realmente suceda, el vector añadido debe ser en algún sentido \emph{ajeno} a los del conjunto generador original. En la siguiente sección, veremos algunas definiciones necesarias para precisar esta idea.
\end{Teo}

