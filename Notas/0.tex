\documentclass[notasLineal]{subfiles}
\begin{document}

\section*{Notación}\label{Sec: Notación}
\begin{tcolorbox} \label{Notación}
    \centering
    \begin{tabular}{cc}
    \\
    $\vec{u}, \vec{v}, \vec{w}, ...$ & vectores (elementos de un conjunto vectorial $V$) \\ \\
    $a,b,c, ...$ & escalares (elementos de un campo $K$ que define un espacio vectorial) \\ \\
    $ab$ & producto entre los escalares $a$ y $b$ \\ \\
    $a\vec{u}$ & producto del vector $\vec{u}$ por el escalar $a$\footnote{Algunos textos se refieren a esta operación \textemdash realizada entre un vector y un escalar, y que da como resultado un vector\textemdash\hspace{1.5mm} como \textit{multiplicación escalar} (o \emph{scalar multiplication}, en inglés); sin embargo, es fácil que esta operación se confunda con la de \textit{producto escalar}, que da como resultado un escalar. Debemos tener esto en mente cuando leamos otros textos de álgebra lineal, tanto en español como en inglés.} \\ \\
    $\overline{a}$ & complejo conjugado del escalar $a$ \\ \\
    $(x_1,...\hspace{0.5mm},x_n) $ & $n$-tupla como coordenada \\ \\
    $\begin{pmatrix}x_1&...&x_n\end{pmatrix}^T$ & $n$-tupla como vector columna \\ \\
    $V + W$ & suma de los espacios vectoriales $V$ y $W$ \\ \\
    $V \oplus W$ & suma directa de los espacios vectoriales $V$ y $W$ \\ \\
    $\langle G \rangle $ & espacio vectorial generado por el conjunto vectorial $G$ \\ \\
    $l.i.$ & linealmente independiente \\ \\
    $l.d.$ & linealmente dependiente \\ \\
    $\text{dim}(V)$ & dimensión del espacio vectorial $V$ \\ \\
    $\langle\vec{u},\vec{v}\rangle$ & producto escalar del vector $\vec{u}$ con el vector $\vec{v}$ \\ \\
    $u_i v_i \equiv \sum_{i=1}^n u_i v_i$ & \textit{notación de Einstein} para la suma sobre un índice repetido \\ \\
    $\vec{u}\perp\vec{v}$ & ortogonalidad entre los vectores $\vec{u}$ y $\vec{v}$ \\ \\
    $\vec{P}_{\vec{v}}(\vec{u})$ & proyección vectorial del vector $\vec{u}$ sobre el vector $\vec{v}$ \\ \\
    $||\vec{u}||$ & norma del vector $\vec{u}$ \\ \\
    $\hat{u}$ & vector normal \\ \\
    %$\mathbf{a}\times\mathbf{b}$ & producto vectorial (cruz) de dos vectores $\mathbf{a},\mathbf{b}\in\mathbb{R}^3$ \\ \\
    %$\mathbf{a}\cdot\mathbf{b}\times\mathbf{c}$ & triple producto escalar entre tres vectores $\mathbf{a},\mathbf{b},\mathbf{c}\in\mathbb{R}^3$. \\ \\
    \end{tabular}
\end{tcolorbox}

\end{document}
