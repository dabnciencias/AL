\documentclass[a4paper]{article}

\usepackage[margin=1.5cm]{geometry}
\usepackage{amsmath,amsthm,amssymb}
\usepackage[spanish,es-tabla]{babel}
\decimalpoint
\usepackage[T1]{fontenc}
\usepackage[utf8]{inputenc}
\usepackage{lmodern}
\usepackage[hyphens]{url}
\usepackage{graphicx}
\graphicspath{ {images/} }
\usepackage{tcolorbox}
\setcounter{section}{-1}
\usepackage{tabularx}
\usepackage{multirow}
\usepackage{hyperref}
\usepackage{braket}
\usepackage{tikz}
\usepackage{pgfplots}
\usepackage{epigraph}
\usetikzlibrary{babel}
\hypersetup{
    colorlinks=true,
    linkcolor=red,
    filecolor=magtenta,
    urlcolor=orange,
}
\newenvironment{theorem}[2][Theorem]{\begin{trivlist}
\item[\hskip \labelsep {\bfseries #1}\hskip \labelsep {\bfseries #2.}]}{\end{trivlist}}
\newenvironment{teorema}[2][Teorema]{\begin{trivlist}
\item[\hskip \labelsep {\bfseries #1}\hskip \labelsep {\bfseries #2.}]}{\end{trivlist}}
\newenvironment{lemma}[2][Lemma]{\begin{trivlist}
\item[\hskip \labelsep {\bfseries #1}\hskip \labelsep {\bfseries #2.}]}{\end{trivlist}}
\newenvironment{exercise}[2][Exercise]{\begin{trivlist}
\item[\hskip \labelsep {\bfseries #1}\hskip \labelsep {\bfseries #2.}]}{\end{trivlist}}
\newenvironment{problem}[2][Problem]{\begin{trivlist}
\item[\hskip \labelsep {\bfseries #1}\hskip \labelsep {\bfseries #2.}]}{\end{trivlist}}
\newenvironment{question}[2][Question]{\begin{trivlist}
\item[\hskip \labelsep {\bfseries #1}\hskip \labelsep {\bfseries #2.}]}{\end{trivlist}}
\newenvironment{corollary}[2][Corollary]{\begin{trivlist}
\item[\hskip \labelsep {\bfseries #1}\hskip \labelsep {\bfseries #2.}]}{\end{trivlist}}
\newenvironment{corolario}[2][Corolario]{\begin{trivlist}
\item[\hskip \labelsep {\bfseries #1}]}{\end{trivlist}}
\newenvironment{solution}{\begin{proof}[Solution]}{\end{proof}}

\begin{document}
\title{Álgebra Lineal \\ Grupo 3044, 2020-II \\ Examen parcial 4 (tarea examen) \\ Fecha de entrega: martes 12 de mayo, 16:00}
\date{}
\maketitle

\epigraph{``\textit{Los recursos, los contactos, el dinero, los amigos... todas esas cosas vienen y van. Lo único que realmente tienes en la vida es tu educación.}''}{\textemdash mi abuelo}

\epigraph{``\textit{Más vale un minuto de pend**** que una vida en duda.}''}{\textemdash mi otro abuelo}

\vspace{5mm}
\textbf{1.} Considera la matriz \[
    M = \begin{pmatrix}0&1&0\\a&0&1\\0&1&0\end{pmatrix}
.\] Encuentra para qué valores de $a$:
\begin{itemize}
    \item $M$ tiene sólo un eigenvalor,
    \item $M$ tiene tres eigenvalores reales,
    \item $M$ tiene tres eigenvalores complejos.
\end{itemize}
Determina si la matriz $M$ es diagonalizable o no para cada uno de estos tres casos y diagonalízala cuando sea posible (en términos del parámetro $a$). (2 ptos.)

\vspace{5mm}
\textbf{2.} Decimos que dos matrices $A$ y $B$ son \emph{simultáneamente diagonalizables} si existe una base $\beta$ tal que $[A]_{\beta}^{\beta}$ y $[B]_{\beta}^{\beta}$ sean matrices diagonales. Sean \[
\sigma_0=\begin{pmatrix}1&0\\0&1\end{pmatrix}, \sigma_1=\begin{pmatrix}0&1\\1&0\end{pmatrix}, \sigma_2=\begin{pmatrix}0&-i\\i&0\end{pmatrix}, \sigma_3=\begin{pmatrix}1&0\\0&-1\end{pmatrix}
.\] Encuentra los eigenvalores e eigenvectores de $\sigma_1$ y $\sigma_2$  y diagonalízalas. Después, responde: ¿cuáles de estas matrices $\sigma_i$ son simultáneamente diagonalizables y por qué? (2 ptos.)

\vspace{5mm}
\textbf{3.} Diagonaliza la matriz $$\begin{pmatrix} \frac{31}{9} & \frac{2}{3} & \frac{4}{9} & \frac{4}{9} \\ \\ \frac{31}{9} & \frac{23}{3} & \frac{13}{9} & \frac{20}{9} \\ \\ -\frac{10}{9} & -\frac{2}{3} & \frac{53}{9} & -\frac{14}{9} \\ \\ -1 & -2 & -3 & 3 \end{pmatrix}  .$$ (2 ptos.)

\vspace{5mm}
\textbf{4.} Sea $V$ un espacio vectorial de dimensión finita $n$ y sea $T:V\to V$ un operador lineal con eigenvalores $\lambda_i$. Demuestra que $V$ se puede descomponer como  una suma directa de los eigenespacios $E_{\lambda_i}$ de $T$ si y sólo si $T$ es diagonalizable. (2 ptos.)

\vspace{5mm}
\textbf{5.} Sea $d_i$ el $i$-ésimo dígito de tu número de cuenta. Considera el siguiente sistema lineal de ecuaciones diferenciales: $$x_1'(t) = d_2(x_2(t)) - d_3,$$ $$x_2'(t)= -d_8(x_1(t)) + d_7.$$ \noindent Argumenta si a través de un cambio de base es posible desacoplar este sistema totalmente; si no es posible, modifica alguna(s) constante(s) $d_i$ para que sí lo sea. Luego, encuentra la solución al sistema y grafícala. (Nota: tal vez necesites hacer un cambio de variable del tipo $y_i(t)=a_i x_i(t)+b_i$ antes de desacoplar el sistema, pero deberás deshacer este cambio al final para expresar la solución en términos de $x_1(t)$ y $x_2(t)$.)

Suponiendo que las funciones $x_1(t)$ y $x_2(t)$ modelan las densidades poblacionales de dos agentes biológicos distintos en un ecosistema cerrado, ¿qué nos dice la solución al sistema sobre la dinámica poblacional entre estos dos agentes? (2 ptos.)

\vspace{5mm}
\textbf{Extra:} Ve el video titulado \href{https://www.youtube.com/watch?v=PFDu9oVAE-g&list=PLZHQObOWTQDPD3MizzM2xVFitgF8hE_ab&index=14}{Eigenvectors and eigenvalues | Essence of linear algebra, chapter 14} del canal de YouTube ``3Blue1Brown'', repasando los temas del módulo pausadamente, y resuelve el ejercicio que aparece al final. (1 pto. extra)*

\end{document}
