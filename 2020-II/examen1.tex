\documentclass[a4paper]{article}

\usepackage[margin=1.5cm]{geometry}
\usepackage{amsmath,amsthm,amssymb}
\usepackage[spanish,es-tabla]{babel}
\decimalpoint
\usepackage[T1]{fontenc}
\usepackage[utf8]{inputenc}
\usepackage{lmodern}
\usepackage[hyphens]{url}
\usepackage{graphicx}
\graphicspath{ {images/} }
\usepackage{tcolorbox}
\setcounter{section}{-1}
\usepackage{tabularx}
\usepackage{multirow}
\usepackage{hyperref}
\usepackage{braket}
\usepackage{tikz}
\usepackage{pgfplots}
\usepackage{epigraph}
\usetikzlibrary{babel}
\hypersetup{
    colorlinks=true,
    linkcolor=red,
    filecolor=magtenta,
    urlcolor=orange,
}
\newenvironment{theorem}[2][Theorem]{\begin{trivlist}
\item[\hskip \labelsep {\bfseries #1}\hskip \labelsep {\bfseries #2.}]}{\end{trivlist}}
\newenvironment{teorema}[2][Teorema]{\begin{trivlist}
\item[\hskip \labelsep {\bfseries #1}\hskip \labelsep {\bfseries #2.}]}{\end{trivlist}}
\newenvironment{lemma}[2][Lemma]{\begin{trivlist}
\item[\hskip \labelsep {\bfseries #1}\hskip \labelsep {\bfseries #2.}]}{\end{trivlist}}
\newenvironment{exercise}[2][Exercise]{\begin{trivlist}
\item[\hskip \labelsep {\bfseries #1}\hskip \labelsep {\bfseries #2.}]}{\end{trivlist}}
\newenvironment{problem}[2][Problem]{\begin{trivlist}
\item[\hskip \labelsep {\bfseries #1}\hskip \labelsep {\bfseries #2.}]}{\end{trivlist}}
\newenvironment{question}[2][Question]{\begin{trivlist}
\item[\hskip \labelsep {\bfseries #1}\hskip \labelsep {\bfseries #2.}]}{\end{trivlist}}
\newenvironment{corollary}[2][Corollary]{\begin{trivlist}
\item[\hskip \labelsep {\bfseries #1}\hskip \labelsep {\bfseries #2.}]}{\end{trivlist}}
\newenvironment{corolario}[2][Corolario]{\begin{trivlist}
\item[\hskip \labelsep {\bfseries #1}]}{\end{trivlist}}
\newenvironment{solution}{\begin{proof}[Solution]}{\end{proof}}

\begin{document}
\title{Álgebra Lineal \\ Grupo 3044, 2020-II \\ Examen parcial 1}
\date{}
\maketitle

\epigraph{``\textit{Sorprenderse, extrañarse, es comenzar a entender.}''}{\textemdash José Ortega y Gasset, \\ filósofo español}

Número de cuenta:

\vspace{5mm}

\textbf{1.} Sea $A=\{f:\mathbb{R}\to\mathbb{R}\mid |\int_{-\infty}^{\infty} f(x)\hspace{1mm} dx| < \infty \}.$ Demuestra que $A$ sobre $\mathbb{R}$ forma un espacio vectorial. (2.5 ptos.)

\vspace{5mm}

\textbf{2.} Sea $F(\mathbb{R},\mathbb{R})$ el espacio vectorial formado por todas las funciones $f:\mathbb{R}\to \mathbb{R}$. Sean $P=\{f:\mathbb{R}\to \mathbb{R}\mid f(-x) = f(x)\}$ e $I=\{f:\mathbb{R}\to \mathbb{R}\mid f(-x)=-f(x)\}$. Demuestra que $P$, $I$ y $P\cap I$ son subespacios vectoriales de $F(\mathbb{R},\mathbb{R})$. Escribe explícitamente a $P\cap I$. (2.5 ptos.)

\vspace{5mm}

\textbf{3.} Demuestra que si $O=\{\mathbf{o}_1,\mathbf{o}_2,...\hspace{1mm} , \mathbf{o}_n\}$ es un conjunto de vectores ortogonales, entonces es linealmente independiente. (2.5 ptos.)

\vspace{5mm}

\textbf{4.} Sea $(V,K)$ un espacio vectorial con $L_1\subset V, L_2\subset V$. Demuestra que si $L_1\subseteq L_2\implies\langle L_1 \rangle \subseteq \langle L_2 \rangle.$ ¿La implicación en el sentido contrario también será válida? Argumenta. (2.5 ptos.)

\vspace{5mm}

\textbf{Extra:} Sea $\mathbf{v}$ un vector no nulo del espacio vectorial complejo $\mathbb{C}$. ¿Cómo puedes interpretar geométricamente el producto del vector $\mathbf{v}$ por el escalar $\frac{1}{i}\in\mathbb{C}$ en el plano complejo? Argumenta y da un ejemplo. (0.5 ptos. extra.)*

\end{document}
