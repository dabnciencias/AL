\documentclass[a4paper]{article}

\usepackage[margin=1.5cm]{geometry}
\usepackage{amsmath,amsthm,amssymb}
\usepackage[spanish,es-tabla]{babel}
\decimalpoint
\usepackage[T1]{fontenc}
\usepackage[utf8]{inputenc}
\usepackage{lmodern}
\usepackage[hyphens]{url}
\usepackage{graphicx}
\graphicspath{ {images/} }
\usepackage{tcolorbox}
\setcounter{section}{-1}
\usepackage{tabularx}
\usepackage{multirow}
\usepackage{hyperref}
\usepackage{braket}
\usepackage{tikz}
\usepackage{enumitem}
\usepackage{pgfplots}
\usepackage{epigraph}
\usetikzlibrary{babel}
\hypersetup{
    colorlinks=true,
    linkcolor=red,
    filecolor=magtenta,
    urlcolor=orange,
}
\newenvironment{theorem}[2][Theorem]{\begin{trivlist}
\item[\hskip \labelsep {\bfseries #1}\hskip \labelsep {\bfseries #2.}]}{\end{trivlist}}
\newenvironment{teorema}[2][Teorema]{\begin{trivlist}
\item[\hskip \labelsep {\bfseries #1}\hskip \labelsep {\bfseries #2.}]}{\end{trivlist}}
\newenvironment{lemma}[2][Lemma]{\begin{trivlist}
\item[\hskip \labelsep {\bfseries #1}\hskip \labelsep {\bfseries #2.}]}{\end{trivlist}}
\newenvironment{exercise}[2][Exercise]{\begin{trivlist}
\item[\hskip \labelsep {\bfseries #1}\hskip \labelsep {\bfseries #2.}]}{\end{trivlist}}
\newenvironment{problem}[2][Problem]{\begin{trivlist}
\item[\hskip \labelsep {\bfseries #1}\hskip \labelsep {\bfseries #2.}]}{\end{trivlist}}
\newenvironment{question}[2][Question]{\begin{trivlist}
\item[\hskip \labelsep {\bfseries #1}\hskip \labelsep {\bfseries #2.}]}{\end{trivlist}}
\newenvironment{corollary}[2][Corollary]{\begin{trivlist}
\item[\hskip \labelsep {\bfseries #1}\hskip \labelsep {\bfseries #2.}]}{\end{trivlist}}
\newenvironment{corolario}[2][Corolario]{\begin{trivlist}
\item[\hskip \labelsep {\bfseries #1}]}{\end{trivlist}}
\newenvironment{solution}{\begin{proof}[Solution]}{\end{proof}}

\begin{document}
\title{Álgebra Lineal \\ Grupo 3003, 2021-I \\ Examen parcial 3 (tarea examen) \\ Fecha límite de entrega: domingo 13 de diciembre, 23:59 hrs.}
\date{}
\maketitle

\epigraph{``La ciencia, más que un cúmulo de conocimientos, es una forma de pensar.''}{\textemdash Carl Sagan, \\ astrofísico estadounidense}

Para quien sea que quiera dedicarse a la ciencia, es indispensable aprender a leer artículos científicos de forma crítica, estudiarlos, digerirlos, extraer las ideas más importantes y, en la medida de lo posible, reproducir los resultados presentados. Ese es el principal objetivo del presente examen.

\vspace{3mm}
Instrucciones para resolver el examen: 

\begin{itemize}
    \item Descarguen (individualmente) el artículo \href{https://arxiv.org/abs/2003.11371}{``Mathematical Modeling of Epidemic Diseases; A Case Study of the COVID-19 Coronavirus''} (sic.) de R. Sameni, disponible en \href{https://arxiv.org}{arXiv}.
    \item Como introducción, lean el resumen (\emph{\textbf{Abstract}}) y las secciones \emph{I.}, \emph{II.A}, y el primer párrafo de la sección \emph{II.D}, subrayando (individualmente) las partes que les parezcan más importantes\footnote{Si pueden y quieren, también pueden agregar anotaciones.}.
    \item La parte central del examen estará enfocada en que \textbf{corroboren los cálculos de la sección} \emph{III.E} (en equipo); sin embargo, para entender qué están haciendo, tendrán que leer y discutir (y releer y rediscutir) toda la sección \emph{III.} En particular, para los cálculos, deberán partir del sistema de ecuaciones (27) y, utilizando la aproximación descrita al inicio de la sección \emph{III.E}, llegar al sistema lineal de ecuaciones diferenciales ordinarias autónomas (34); luego, deberán verificar si las ecuaciones en (36), (37), (38), (40), (41) y (43) son correctas\footnote{Aquí les va una mala noticia: los artículos científicos a menudo contienen errores que hacen que sus resultados no sean reproducibles \textemdash inclusive aquellos escritos por personas de renombre y publicados en revistas de alto prestigio internacional\textemdash; por ello, es importante siempre examinarlos de forma crítica. La buena noticia es que muchos artículos contienen información suficiente para poder verificar que el trabajo esté bien hecho, pero hay que ``echarse la talacha''; ése es su trabajo.}. (4 ptos.)
    \item ¿Qué justifica la ecuación (39)? Demuéstrenlo para el caso general\footnote{Pueden encontrar el planteamiento como el Teorema 4.40 del Poole (pág. 342).}. (3 ptos.)
    \item ¿Los resultados que aparecen como \textbf{Result 2}, \textbf{Result 3} y \textbf{Result 4} están respaldados por lo cálculos que hicieron? Argumenten su respuesta. (1 ptos)
    \item Lean la sección \emph{IV.} y verifiquen si las ecuaciones (48), (49) y (50) son correctas. (2 ptos.)
    \item Extra: reproduzcan las subfiguras de la Fig. 8, utilizando su \emph{software} de preferencia. (2 ptos.)
\end{itemize}

Para la entrega deberán, como de costumbre, juntar el trabajo grupal en un solo documento sin páginas rotadas. Una sola persona por equipo deberá enviarle dicho documento \textbf{a Javier} antes de la fecha límite de entrega. Además, individualmente, deberán enviarme \textbf{a mí (Diego)} su archivo con el artículo subrayado/comentado en las secciones indicadas, nombrándolo con su número de cuenta \textemdash por ejemplo, ``123456789.pdf''\textemdash  \ antes de la fecha límite de entrega.

\end{document}
