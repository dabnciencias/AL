\documentclass[12pt]{article}
 
\usepackage[margin=1.5cm]{geometry} 
\usepackage{amsmath,amsthm,amssymb}
\usepackage[spanish,es-tabla]{babel}
\decimalpoint
\usepackage[T1]{fontenc}
\usepackage[utf8]{inputenc}
\usepackage{lmodern}
\usepackage{graphicx}
\graphicspath{ {images/} }
\usepackage{tcolorbox}
\setcounter{section}{-1}
\usepackage{tabularx}
\usepackage{multirow}
\usepackage{hyperref}
\usepackage{braket}
\hypersetup{
    colorlinks=true,
    linkcolor=red,
    filecolor=magenta,      
    urlcolor=orange,
}
\newcommand{\N}{\mathbb{N}}
\newcommand{\Z}{\mathbb{Z}}
\newenvironment{theorem}[2][Theorem]{\begin{trivlist}
\item[\hskip \labelsep {\bfseries #1}\hskip \labelsep {\bfseries #2.}]}{\end{trivlist}}
\newenvironment{teorema}[2][Teorema]{\begin{trivlist}
\item[\hskip \labelsep {\bfseries #1}\hskip \labelsep {\bfseries #2.}]}{\end{trivlist}}
\newenvironment{lemma}[2][Lemma]{\begin{trivlist}
\item[\hskip \labelsep {\bfseries #1}\hskip \labelsep {\bfseries #2.}]}{\end{trivlist}}
\newenvironment{exercise}[2][Exercise]{\begin{trivlist}
\item[\hskip \labelsep {\bfseries #1}\hskip \labelsep {\bfseries #2.}]}{\end{trivlist}}
\newenvironment{problem}[2][Problem]{\begin{trivlist}
\item[\hskip \labelsep {\bfseries #1}\hskip \labelsep {\bfseries #2.}]}{\end{trivlist}}
\newenvironment{question}[2][Question]{\begin{trivlist}
\item[\hskip \labelsep {\bfseries #1}\hskip \labelsep {\bfseries #2.}]}{\end{trivlist}}
\newenvironment{corollary}[2][Corollary]{\begin{trivlist}
\item[\hskip \labelsep {\bfseries #1}\hskip \labelsep {\bfseries #2.}]}{\end{trivlist}}
\newenvironment{corolario}[2][Corolario]{\begin{trivlist}
\item[\hskip \labelsep {\bfseries #1}]}{\end{trivlist}}
\newenvironment{solution}{\begin{proof}[Solution]}{\end{proof}}
 
\begin{document}
 
\title{Planeación de curso de Álgebra Lineal \\ (2020-II)}
\author{Diego Alberto Barceló Nieves\\
Lic. en Física Biomédica (en proceso de titulación) \\ Universidad Nacional Autónoma de México}

\maketitle

\section{Introducción}

La presente planeación del curso de Álgebra Lineal para la Licenciatura en Física Biomédica fue realizada tomando en cuenta el plan de estudios existente para la materia\footnote{El cual puede ser consultado en la página \url{http://www.fciencias.unam.mx/asignaturas/1330.pdf}.}, así como la retroalimentación del mismo por parte de estudiantes de las dos primeras generaciones de la licenciatura.

Al inicio del curso se procurará hacer un mayor énfasis en las estrechas relaciones existentes entre el álgebra lineal y el cálculo vectorial ya que, actualmente, se sugiere que las materias de Álgebra Lineal y Cálculo Avanzado se cursen simultáneamente. Además, durante todo el curso, se buscará sentar bases teóricas sólidas para diversas aplicaciones que serán útiles en materias posteriores del plan de estudios, tales como Ecuaciones Diferenciales (e.g., matrices, determinantes, eigenvalores y eigenvectores para aplicarse en sistemas de ecuaciones diferenciales), Matemáticas Avanzadas (e.g., bases ortogonales y ortonormales de funciones para ser aplicado en funciones especiales), Introducción a la Física Cuántica/Mecánica Cuántica (e.g., espacios de Hilbert, espacios duales y operadores hermitianos para aplicarse en el formalismo de la Mecánica Cuántica) e Imagenología Biomédica (e.g., matrices, transformaciones lineales y transformaciones rígidas para ser aplicado a procesamiento digital de imágenes).


\subsection{Objetivo general del curso}

Estudiar los espacios vectoriales, las transformaciones lineales y sus representaciones matriciales para su posterior aplicación en ciertas áreas de las matemáticas, la física y la biomedicina.

\subsection{Forma de evaluación}

Debido a la gran cantidad de temas que se deberán ver en el curso, la evaluación será por medio de tareas semanales (que se entregarán todos los días lunes, a partir del segundo lunes de clases) y 5 exámenes parciales, los cuales se realizarán siempre en lunes (después de la entrega de la tarea correspondiente) y durarán una hora. Los porcentajes correspondientes a estos dos rubros se definirán con el grupo al inicio del curso. Algunas tareas incluirán ejercicios extra para subir calificación, y los exámenes con calificación menor a 6 podrán ser corregidos por l@s estudiantes y reentregados el siguiente lunes para subir calificación. Para definir la calificación final del curso, en los casos donde la calificación obtenida sea muy cercana a la calificación siguiente, se tomará en cuenta el esfuerzo extra realizado por cada estudiante durante el semestre mediante la entrega de correcciones de exámenes y/o ejercicios extra en tareas para definir un posible cambio.



\newpage
\subsection{Cronograma tentativo del curso}

\begin{table}[ht]
    \centering
    \begin{tabularx}{\linewidth}{|c|X|}
    \hline Semana & Temas \\
    \hline 1 & Campos, espacios vectoriales, interpretación geométrica de las operaciones de los espacios vectoriales y subespacios vectoriales \\
    \hline 2 & Producto escalar (punto), norma, proyecciones y ortogonalidad, producto vectorial (cruz)* y triple producto escalar* \\
    \hline 3 & Combinaciones lineales, espacio generado y conjunto generador, dependencia e independencia lineal \\
    \hline 4 & Bases, dimensión, ortogonalización y ortonormalización \\
    \hline 5 & Definición de transformación lineal y espacio de transformaciones lineales, núcleo e imagen de una transformación lineal \\
    \hline 6 & Composición de transformaciones lineales, transformación inversa y representación matricial de una transformación lineal \\
    \hline 7 & Espacios vectoriales de matrices, matriz de proyección y matriz de cambio de base \\
    \hline 8 & Definición y propiedades del determinante y la traza de una matriz \\
    \hline 9 & Matriz inversa e invertibilidad de matrices \\
    \hline 10 & Definición y propiedades de valores propios y vectores propios, polinomio característico \\
    \hline 11 & Propiedades de las transformaciones simétricas, subespacios invariantes y diagonalización \\
    \hline 12 & Definición y representaciones de un operador lineal, definiciones de operador adjunto, funcional y espacio dual \\
    \hline 13 & Definición y propiedades de operadores hermitianos \\
    \hline 14 & Descomposición espectral de transformaciones simétricas \\
    \hline 15 & Sistemas de ecuaciones lineales algebráicas y diferenciales, bases ortogonales y ortonormales de funciones \\
    \hline 16 & Espacios de Hilbert, producto tensorial y conmutadores* \\
    \hline
    \end{tabularx}
    \caption{Cronograma semanal tentativo del curso de Álgebra Lineal. Los temas marcados con asterisco (*) no se planean ver en clase, pero sí se mencionarán en las notas y aparecerán ejercicios relacionados a ellos en las tareas, los cuales tendrán valor de \emph{puntos extra}.}
    \label{Cronograma}
\end{table}{}

\subsection{Bibliografía recomendada del curso} \label{Bibliografía}

\begin{itemize}
    \item S. H. Friedberg, \textit{Linear Algebra}, 4ta ed. - es el texto básico para este tipo de cursos.
    \item S. Lang, \textit{Introduction to Linear Algebra}, 2da ed. (Springer, 1986, EUA) - otro texto básico que cubre la muchos de los temas del curso y complementa el Friedberg.
    \item S. Lang, \textit{Linear Algebra}, 3a ed. (Springer, 1987, EUA) - no es tan básico como \textit{Introduction to Linear Algebra}, pero sirve para \textit{repasar} temas básicos y aprender algunos más avanzados.
    \item M. Vujičić, \textit{Linear Algebra Thoroughly Explained} (Springer, 2008, EUA) - este texto va de lo más básico a lo más avanzado y servirá para ver algunos temas importantes con especial detenimiento.
    \item D. Poole, \textit{Linear Algebra: A Modern Introduction}, 4ta ed. (Cengage Learning, 2015, EUA) - este texto básico ve los temas a través de diversos ejemplos y aplicaciones; sirve para las personas que quieran ver algunas aplicaciones de los conceptos al mismo tiempo que los aprenden.
\end{itemize}{}

Les sugiero que hojeen \textit{todos} los libros recomendados al inicio del curso, y que consulten los de su agrado constantemente durante el mismo, o bien, busquen otros que les sirvan mejor para aprender.


\newpage
\section{Campos, espacios vectoriales, interpretación geométrica de las operaciones de los espacios vectoriales y subespacios vectoriales}

\vspace{1cm}

\begin{tcolorbox} \label{1:Notación}
\underline{Notación:}

\centering
\begin{tabular}{cc}
    \\
    $\mathbf{u}, \mathbf{v}, \mathbf{w}, ...$ & vectores (elementos de un conjunto vectorial $V$) \\ \\
    $a,b,c, ...$ & escalares (elementos de un campo $K$, que define un espacio vectorial) \\ \\
    $ab$ & producto entre los escalares $a$ y $b$ \\ \\
    $a\mathbf{u}$ & producto del vector $\mathbf{u}$ por el escalar $a$\footnote{Algunos textos se refieren a esta operación \textemdash realizada entre un vector y un escalar, y que da como resultado un vector\textemdash\hspace{1.5mm} como \textit{multiplicación escalar} (o \emph{scalar multiplication}, en inglés); sin embargo, es fácil que esta operación se confunda con la de \textit{producto escalar}, que da como resultado un escalar. Debemos tener esto en mente cuando leamos otros textos de álgebra lineal, tanto en español como en inglés.} \\ \\
    $\begin{bmatrix}x_1 \\ x_2 \\ ... \\ x_n\end{bmatrix}\equiv\begin{bmatrix}x_1&x_2&...&x_n\end{bmatrix}^T$ & vector como columna de matriz \\ \\
    $V + W$ & suma de los espacios vectoriales $V$ y $W$ \\ \\
    $V \oplus W$ & suma directa de los espacios vectoriales $V$ y $W$ \\
\end{tabular}
\end{tcolorbox}


\newpage
\subsection{Campos}
Uno de los conceptos más fundamentales del álgebra lineal es el de espacio vectorial; sin embargo, para definir formalmente un espacio vectorial se requiere de una estructura algebráica conocida como campo \textemdash el ejemplo más común siendo el \emph{campo de los números reales}. Esta estructura quizá la viste explícitamente en tu curso de Álgebra y/o implícitamente en tu curso de Cálculo I (a través de los \textit{axiomas de campo} \textemdash o de cuerpo\textemdash\hspace{1.5mm} \textit{para los números reales}); sin embargo, mencionaremos su definición y los dos ejemplos de campos que más utilizaremos durante este curso (el campo de los números reales y el de los complejos) antes de definir los espacios vectoriales.

\subsubsection{Definición de campo} \label{Def:Campo}

\begin{tcolorbox}
\underline{Def.} Un \emph{campo} es un conjunto $K$ con dos operaciones (llamadas \emph{adición} o \emph{suma} y \emph{multiplicación}) que cumplen las propiedades siguientes:

\begin{center}
\begin{tabular}{lr}
    \\
    $\forall\hspace{1.5mm}a,b\in K\hspace{2.5mm} a+b=b+a;\hspace{1.5mm} ab = ba
    $ & Conmutatividad \\ \\
    $\forall\hspace{1.5mm}a,b,c\in K\hspace{2.5mm}a+(b+c)=(a+b)+c;\hspace{1.5mm}a(bc)=(ab)c$ & Asociatividad \\ \\
    $\exists\hspace{1.5mm} 0,1\in K$ t.q. $a+0=a$, $1a=a\hspace{2.5mm} \forall\hspace{1.5mm}a \in K$ & Elementos idetidad (neutros) \\ \\
    $\forall\hspace{1.5mm}a\in K\hspace{2.5mm}\exists\hspace{0.5mm}-a\in K$ t.q. $a + (-a) = 0$ & Elemento inverso de la suma \\ \\
    $\forall\hspace{1.5mm}a\neq0\in K \hspace{2.5mm} \exists\hspace{1.5mm}a^{-1}\in K$ t.q. $a(a^{-1})= 1$ & Elemento inverso de la multiplicación \\ \\
    $\forall\hspace{1.5mm}a,b,c\in K\hspace{2.5mm}a(b+c) = ab+ac$ & Distributividad.\\ \\
\end{tabular}
\end{center}

\noindent A las propiedades anteriores también se les conoce como \emph{axiomas de campo}\footnote{Nótese que la última de ellas (distributividad) es la única que \emph{combina} ambas operaciones definidas en el campo}. A pesar de que la definición de campo incluya un conjunto $K$ y dos operaciones que cumplen las propiedades anteriormente mencionadas, por simplicidad, se suele denotar a todo el campo como $K$.
\end{tcolorbox}

\subsubsection{El campo $\mathbb{R}$}

El conjunto de los números reales $\mathbb{R}$ junto con las operaciones de suma y multiplicación (que aprendimos de forma intuitiva durante nuestra educación básica) cumplen todas las propiedades enlistadas en la sección \ref{Def:Campo}, ya que dichas operaciones son conmutativas, asociativas y cumplen la propiedad de distributividad. El elemento identidad (neutro) de la suma es $0\in\mathbb{R}$ y el de la multiplicación es $1\in\mathbb{R}$. Para todo número $a\neq0\in\mathbb{R}$, el elemento inverso de la multiplicación es $a^{-1} = \frac{1}{a}\in\mathbb{R}$. Al conjunto $\mathbb{R}$ junto con estas dos operaciones se le conoce como el \emph{campo de los números reales} o, simplemente, el \emph{campo real}.

\subsubsection{El campo $\mathbb{C}$} \label{Ejem:Campo_complejo}

El conjunto de los números complejos se define como $\mathbb{C} \equiv \{a+ib\hspace{1.5mm}|\hspace{1.5mm}a,b\in\mathbb{R}\}$, donde $i\equiv+\sqrt{-1}$. Definiendo la suma entre números complejos como $(a+ib)+(c+id)\equiv(a+c) + (b+d)i$ y la multiplicación entre números complejos como\footnote{Nótese que esta definición es simplemente el resultado de desarrollar $(a+ib)(c+id)$ como un producto de dos binomios.} $(a+ib)(c+id)\equiv (ac-bd) + (ad+bc)i$, podemos comprobar que estas dos operaciones junto con el conjunto $\mathbb{C}$ forman un campo. A este campo se le conoce como el \emph{campo de los números complejos} o \emph{campo complejo}.

\vspace{3mm}

Observemos que prácticamente todas las operaciones que realizamos en nuestra vida cotidiana como calcular fechas y tiempos, armar presupuestos, dar cambio, proyectar ahorros, aproximar áreas, repartir comida, etc., toman lugar en un campo. Por ejemplo, podríamos decir que la operación mental necesaria para repartir un pastel sucede en el campo de los números racionales. Es decir, las ideas intuitivas que nos formamos durante la educación básica de que la suma siempre debe ser conmutativa y asociativa \textemdash al igual que la multiplicación\textemdash\hspace{0.5mm}, que existe la resta y la división, que el $0$ y el $1$ son números \emph{especiales} y que siempre se cumple la propiedad de distributividad, son un \emph{hecho} para cualquier estructura de campo. Sin embargo, estas mismas ideas intuitivas \emph{no son un hecho para otros tipos de estructuras algebráicas} \textemdash algunos de las cuales veremos más adelante\textemdash\hspace{0.5mm}, ¡así que no se vayan con la finta!

\newpage
\subsection{Espacios vectoriales}

Un espacio vectorial es una estructura algebráica abstracta que cumple una serie de propiedades específicas que veremos en el siguiente apartado. Dicha estructura tiene una gran variedad de aplicaciones en muchas áreas de las matemáticas, la física, la computación y la biomedicina, por lo cual, para arrancar el curso, es vital su comprensión desde un punto de vista teórico.

\subsubsection{Definición de espacio vectorial} \label{Def:Espacio_vectorial}

\begin{tcolorbox}
\underline{Def.} Un \textit{espacio vectorial} sobre un campo\footnote{Ver sección \ref{Def:Campo}.} $K$ es un conjunto $V$ con dos operaciones (llamadas \textit{adición} o \textit{suma vectorial} y \textit{producto de un vector por un escalar}) que satisfacen las siguientes propiedades:

\begin{center}
\begin{tabular}{lr}
    $\forall\hspace{1.5mm} \mathbf{u},\mathbf{v}\in V \hspace{3mm}\exists \hspace{1.5mm} \mathbf{u}+\mathbf{v}\in V$ & Cerradura de la adición \\ \\ \multirow{2}{0.4\textwidth}{$\forall\hspace{1.5mm} \mathbf{v}\in V, a\in K \hspace{3mm}\exists \hspace{1.5mm} a\mathbf{v}\in V$} & \multirow{2}{0.28\textwidth}{Cerradura del producto de un vector por un escalar} \\ \\ \\
    $\forall\hspace{1.5mm} \mathbf{u},\mathbf{v},\mathbf{w}\in V\hspace{3mm}\mathbf{u}+(\mathbf{v}+\mathbf{w})=(\mathbf{u}+\mathbf{v})+\mathbf{w}$  & Asociatividad de la adición\\ \\
    $\forall\hspace{1.5mm} \mathbf{u},\mathbf{v}\in V\hspace{3mm}\mathbf{u}+\mathbf{v}=\mathbf{v}+\mathbf{u}$ & Conmutatividad de la adición \\ \\
    $\exists \hspace{1.5mm} \mathbf{0}\in V$ t.q. $\mathbf{v}+\mathbf{0}=\mathbf{v}\hspace{3mm}\forall\hspace{1.5mm} \mathbf{v} \in V$ & Elemento identidad de la adición (neutro aditivo) \\ \\
    $\forall\hspace{1.5mm}\mathbf{v}\in V \hspace{3mm}\exists\hspace{1.5mm} -\mathbf{v}\in V$ t.q. $\mathbf{v}+(-\mathbf{v})=\mathbf{0}$ & Elemento inverso de la adición (inverso aditivo) \\ \\
    \multirow{2}{0.35\textwidth}{$(ab)\mathbf{v}=a(b\mathbf{v})\hspace{3mm}\forall a,b\in K, \mathbf{v}\in V$} & \multirow{2}{0.47\textwidth}{Compatibilidad del producto de un vector por un escalar con el producto entre escalares} \\ \\ \\
    \multirow{2}{0.4\textwidth}{$\exists\hspace{1.5mm}1\in K$ \hspace{1.5mm} t.q. $\hspace{1.5mm}1\mathbf{v}=\mathbf{v}\hspace{3mm}\forall\hspace{1.5mm} \mathbf{v}\in V$} & \multirow{2}{0.35\textwidth}{Elemento identidad del producto de un vector por un escalar} \\ \\ \\
    \multirow{2}{0.4\textwidth}{$a(\mathbf{v}+\mathbf{w})=a\mathbf{v}+a\mathbf{w}\hspace{3mm}\forall\hspace{1.5mm} \mathbf{v},\mathbf{w}\in V, a\in K$} & \multirow{2}{0.47\textwidth}{Distributividad del producto de un vector por un escalar con respecto a la adición vectorial}  \\ \\ \\
    \multirow{2}{0.4\textwidth}{$(a+b)\mathbf{v}=a\mathbf{v}+b\mathbf{v}\hspace{3mm}\forall\hspace{1.5mm} a,b\in K, \mathbf{v}\in V$} & \multirow{2}{0.47\textwidth}{Distributividad del producto de un vector por un escalar con respecto a la adición escalar.} \\ \\
\end{tabular}
\end{center}

\hspace{2.5mm} A los elementos del campo $a,b \in K$ utilizado para definir el espacio vectorial se les llama \textit{escalares} y los elementos $\mathbf{u},\mathbf{v},\mathbf{w}\in V$ que cumplen todas las propiedades anteriores son llamados \textit{vectores}. A las propiedades anteriores también se les conoce como \textit{axiomas de espacios vectoriales}.

\end{tcolorbox}
 

Partiendo de esta definición, podemos hacer varias observaciones:

\begin{itemize}
    \item La definición matemática de \textit{vectores} como \textit{elementos cualesquiera de un conjunto V que \textemdash junto con un campo $K$\textemdash\hspace{0.5mm} cumplen las propiedades de un espacio vectorial} es muy distinta a la definición de vector como \textit{elemento con magnitud, dirección y sentido (y, más precisamente, que además es invariante bajo rotaciones propias e impropias)} utilizada en algunas áreas de la física, siendo la primera definición más general.
    \item La definición de \textit{espacio vectorial} incluye dos operaciones \textit{nuevas} (con respecto a las operaciones de campo) con una importante diferencia entre ellas: una es sólamente entre los elementos del conjunto $V$ (adición o suma vectorial) y, la otra, entre los elementos del conjunto $V$ y el campo $K$ (producto de un vector por un escalar)\footnote{Más adelante veremos otras operaciones que se pueden definir entre vectores y escalares, pero las dos que hemos visto hasta ahora son las únicas necesarias para \textit{definir} los espacios vectoriales.}. Sin embargo, \emph{ambas dan como resultado un vector en $V$}.
    \item Así como la definición de \textit{campo} incluye un conjunto $K$ con dos operaciones (suma y producto) entre sus elementos que cumplen propiedades específicas, la definición de \textit{espacio vectorial} incluye un conjunto $V$ y un campo $K$ con dos operaciones (suma vectorial y producto de un vector por un escalar) entre sus elementos que cumplen propiedades específicas. Por simplicidad, al campo se le denota como $K$ y al espacio vectorial, como $V$.
    
\end{itemize}{}

Para complementar la discusión al respecto de qué es un vector y apreciar cómo funcionan las operaciones de los espacios vectoriales (suma vectorial y producto de un vector por un escalar) de manera visual, sugiero ver el siguiente video: \url{https://www.youtube.com/watch?v=fNk_zzaMoSs}.

\subsubsection{Ejemplos de espacios vectoriales} \label{Ejem:Espacios_vectoriales}

El producto cartesiano del conjunto de los números reales consigo mismo $\mathbb{R}\times\mathbb{R}$ (o $\mathbb{R}^2$) sobre el campo $\mathbb{R}$ es un espacio vectorial, ya que los elementos de $\mathbb{R}^2$ (conocidos como \textit{pares ordenados} o \textit{2-tuplas}) junto con los de $\mathbb{R}$ satisfacen la definición de la sección \ref{Def:Espacio_vectorial}. Usualmente, en geometría analítica, estos pares ordenados se escriben en forma de \emph{coordenadas} $(x,y)$ con $x,y \in \mathbb{R}$ y tienen una correspondencia uno a uno con \emph{puntos} en el plano cartesiano; sin embargo, en álgebra lineal, cuando hablemos de pares ordenados como \emph{vectores} de $\mathbb{R}^2$, es preferible emplear la notación\footnote{A esto se le conoce como \emph{escribir un vector como una columna de matriz} (ver el recuadro \ref{1:Notación}) y facilita la notación al momento de hacer productos de matrices por la izquierda con vectores, como haremos más adelante en el curso.} $\textbf{x}=\begin{bmatrix}x_1\\x_2\end{bmatrix}\in\mathbb{R}^2,\hspace{1.5mm}x_1,x_2\in\mathbb{R}$ (o, equivalentemente, $\begin{bmatrix}x_1&x_2\end{bmatrix}^T$). Estos vectores se pueden representar visualmente mediante una correspondencia uno a uno con \emph{flechas} en el plano cartesiano, las cuales \emph{tienen su cola en el origen} y \emph{su punta en el punto correspondiente a las coordenadas} $(x_1,x_2)$. La suma vectorial se define naturalmente como $\begin{bmatrix}x_1\\x_2\end{bmatrix}+\begin{bmatrix}y_1\\y_2\end{bmatrix}=\begin{bmatrix}x_1+y_1\\x_2+y_2\end{bmatrix}$. El elemento identidad de la suma (neutro aditivo) es el \emph{vector origen} $\mathbf{0}=\begin{bmatrix}0\\0\end{bmatrix}$ y el inverso aditivo de un vector $\mathbf{u}=\begin{bmatrix}u_1\\u_2\end{bmatrix}$ es $\begin{bmatrix}-u_1\\-u_2\end{bmatrix}$, denotado como $-\mathbf{u}$. En este caso, el producto de un vector $\textbf{v}=\begin{bmatrix}v_1\\v_2\end{bmatrix}\in\mathbb{R}^2$ por un escalar $a\in\mathbb{R}$ se define como $a\textbf{v}=a\begin{bmatrix}v_1\\v_2\end{bmatrix} \equiv \begin{bmatrix}av_1\\av_2\end{bmatrix}$ (o $a\begin{bmatrix}v_1&v_2\end{bmatrix}^T \equiv \begin{bmatrix}av_1&av_2\end{bmatrix}^T$) y el elemento de identidad de esta operación es el escalar $1\in\mathbb{R}$.

\vspace{3mm}

El espacio vectorial de $\mathbb{R}^3$ sobre $\mathbb{R}$ se define de manera análoga. Los vectores de $\mathbb{R}^3$ también se pueden representar como flechas que parten del origen de un espacio tridimensional, en una correspondencia uno a uno con las coordenadas del espacio tridimensional. Este tipo de espacios vectoriales se pueden generalizar de manera abstracta (i.e., no visual), como veremos en el siguiente ejemplo.

\vspace{3mm}

El conjunto obtenido al realizar un producto cartesiano de un número entero positivo $n$ de conjuntos $\mathbb{R}$ ($\mathbb{R}\times\mathbb{R}\times...\times\mathbb{R} = \mathbb{R}^n$) sobre el campo $\mathbb{R}$ también es un espacio vectorial. Sus elementos vectoriales son de la forma $\mathbf{v} = \begin{bmatrix}v_1&v_2& ... & v_n\end{bmatrix}^T, \hspace{1.5mm} v_i \in \mathbb{R}, \hspace{1.5mm} i \in \{1,2,...,n\}$ y son conocidos como \textit{n-tuplas}\footnote{Nótese que inclusive en el caso $n=1$ el conjunto $\mathbb{R}$ sobre sí mismo forma un espacio vectorial. Es decir, en este caso, $\mathbb{R}$ funciona como conjunto vectorial y como campo.}. Las operaciones entre los vectores de $\mathbb{R}^n$ y con los escalares en $\mathbb{R}$ se definen de manera análoga al ejemplo de $\mathbb{R}^2$. A este tipo de espacios vectoriales les llamamos \textit{espacios vectoriales reales}, de acuerdo con la definición siguiente:

\begin{tcolorbox}
\underline{Def.} Un \textit{espacio vectorial real} es aquel definido sobre el campo $\mathbb{R}$ (campo real) o, equivalentemente, aquel donde los escalares son números reales.
\end{tcolorbox}{}

\vspace{3mm}

El conjunto de todas las funciones polinomiales de una variable real de grado $n$ (i.e., con regla de correspondencia de la forma $f(x) = c_1 x^1 + c_2 x^2 + ... + c_n x^n, \hspace{1.5mm} c_i \in \mathbb{R}, \hspace{1.5mm} i \in \{1,2,...,n\}$) sobre el campo $\mathbb{R}$ forma un espacio vectorial. Aquí, las definiciones de suma vectorial y de producto de un vector por un escalar se siguen naturalmente de la definición de la suma de funciones $(f+g)(x)\equiv f(x)+g(x)$ y del producto de una función arbitraria $f(x)$ por una función constante $a$, respectivamente, vistas en cálculo. El elemento identidad de la suma vectorial (neutro aditivo) es la función constante cero $f(x)=0\hspace{2.5mm} \forall\hspace{0.5mm}x$ y el inverso aditivo de una función $g(x)$ es $-g(x)$. Observemos que, en este caso, los \textit{vectores} de nuestro espacio vectorial \textit{son funciones} (en particular, en este ejemplo, son funciones polinomiales).

\vspace{3mm}

El conjunto de todas las funciones de una variable real derivables y con derivada continua (i.e., funciones de clase $C^1$) sobre el campo $\mathbb{R}$ forma un espacio vectorial\footnote{En general, el conjunto de funciones de clase $C^n$ sobre el campo $\mathbb{R}$ forma un espacio vectorial.}. Esto probablemente lo viste de manera implícita en tu curso de cálculo diferencial de una variable, cuando viste los teoremas de derivada de una suma/multiplicación/división de funciones (también conocido como \emph{álgebra de derivadas}) para funciones de este tipo. Las operaciones en este espacio vectorial, así como los elementos identidad (neutros) e inversos, se definen de la misma forma que en el ejemplo de las funciones polinomiales.

\vspace{3mm}

El conjunto $\mathbb{C}\times\mathbb{C}$ ($\mathbb{C}^2$) sobre el campo $\mathbb{C}$ también es un espacio vectorial. Sus vectores son de la forma $\begin{bmatrix}a+ib&c+id\end{bmatrix}^T$ con $a,b,c,d\in\mathbb{R}$ e $i\equiv+\sqrt{-1}$ (ya que esto implica que $a+ib, c+id\in\mathbb{C}$, es decir, que sus entradas son complejas). El elemento identidad de la suma vectorial es $\mathbf{0}=\begin{bmatrix}0+i0&0+i0\end{bmatrix}^T$ y el del producto de un vector por un escalar es $1 + i0\in\mathbb{C}$; las operaciones en $\mathbb{C}^2$ y $\mathbb{C}$ se definen como las del ejemplo de $\mathbb{R}^2$ y $\mathbb{R}$. Análogamente, el conjunto $\mathbb{C}^n$ sobre el campo $\mathbb{C}$ es un espacio vectorial: sus vectores tienen $n$ entradas complejas y sus escalares también son complejos. A este tipo de espacio vectorial le llamamos \textit{espacio vectorial complejo}, de acuerdo a la siguiente definición:

\begin{tcolorbox}
\underline{Def.} Un \textit{espacio vectorial complejo} es aquel definido sobre el campo $\mathbb{C}$ (campo complejo) o, equivalentemente, aquel donde los escalares son números complejos.
\end{tcolorbox}{}

Nota: no podemos visualizar los vectores de $\mathbb{R}^n$ con $n>3$, los de $\mathbb{C}^m$ con $m>1$, ni los del conjunto de funciones de clase $C^1$, etc., como \emph{flechas que parten de un mismo origen}\footnote{Es posible visualizar vectores más abstractos de otras formas: \url{https://www.youtube.com/watch?v=zwAD6dRSVyI}.}. Sin embargo, sí podemos hacer operaciones entre estos vectores de manera análoga a como lo haríamos con vectores de $\mathbb{R}^2$ o $\mathbb{R}^3$, por lo cual trabajar en estos espacios \emph{visualizables} puede ayudarnos a generar intuición sobre espacios vectoriales más abstractos.

\vspace{3mm}
Para ver más ejemplos de espacios vectoriales pueden revisar, por ejemplo, \textit{Linear Algebra} de Friedberg, págs. 8-11., \textit{Introduction to Linear Algebra} de Lang, págs. 89-90. o \textit{Linear Algebra: A Modern Introduction} de Poole, págs. 430-432, entre otros.

\subsubsection{Algunos teoremas de espacios vectoriales}

\begin{teorema} 1
Sean $\mathbf{x},\mathbf{y},\mathbf{z}$ vectores de $V$ tales que $\mathbf{x}+\mathbf{z}=\mathbf{y}+\mathbf{z}$, entonces $\mathbf{x}=\mathbf{y}$.

\begin{proof}
Ya que $\mathbf{z}$ es un vector de $V\implies$ $\exists\hspace{2mm} \mathbf{-z}\in V$ tal que $\mathbf{z} + (-\mathbf{z}) = \mathbf{0}$, por la propiedad de existencia de inversos aditivos de los espacios vectoriales. Sumando este elemento $-\mathbf{z}$ a cada lado de la igualdad, tenemos que $$\mathbf{x}+\mathbf{z}=\mathbf{y}+\mathbf{z}\iff\mathbf{x}+\mathbf{z}+ (-\mathbf{z})=\mathbf{y}+\mathbf{z}+ (-\mathbf{z})\iff\mathbf{x}+(\mathbf{z}+ (-\mathbf{z}))=\mathbf{y}+(\mathbf{z}+ (-\mathbf{z}))\iff$$ $$ \mathbf{x}+\mathbf{0}=\mathbf{y}+\mathbf{0}\iff\mathbf{x}=\mathbf{y},$$

\noindent donde en la tercera igualdad se aplicó la propiedad asociativa de la suma vectorial, en la cuarta igualdad se aplicó la propiedad de existencia de los inversos aditivos y, en la última, se aplicó la propiedad de existencia de neutro aditivo.
\end{proof}
A este teorema se le conoce como \emph{Ley de cancelación para la suma vectorial} y con él se puede demostrar la unicidad del nuetro aditivo y de los inversos aditivos.
\end{teorema}

\begin{teorema} 2
Sea $V$ sobre $K$ un espacio vectorial arbitrario con $\mathbf{v}\in V$ y $a\in K$, entonces se verifica que:

\begin{enumerate}
    \item $0\mathbf{v}=\mathbf{0}$
    \item $a\mathbf{0}=\mathbf{0}$
    \item $(-a)\mathbf{v}=-(a\mathbf{v})=a(-\mathbf{v})$
\end{enumerate}

\begin{proof}
\begin{enumerate}
    \item $0\mathbf{v}+0\mathbf{v}=(0+0)\mathbf{v}=0\mathbf{v}=0\mathbf{v}+\mathbf{0}\iff0\mathbf{v}=\mathbf{0}$, donde se aplicaron las propiedades de distributividad del producto de un vector por un escalar con respecto a la adición \emph{escalar}, existencia del neutro aditivo y la ley de cancelación de la suma vectorial (Teorema 1).
    \item $a\mathbf{0}+a\mathbf{0}=a(\mathbf{0}+\mathbf{0})=a\mathbf{0}=a\mathbf{0}+\mathbf{0}\iff a\mathbf{0}=\mathbf{0}$, donde se aplicaron las propiedades de distributividad del producto de un vector por un escalar con respecto a la adición \emph{vectorial}, existencia del neutro aditivo y la ley de cancelación de la suma vectorial.
    \item Por el primer inciso y por distributividad, $\mathbf{0} = (0)\mathbf{v} = (a+(-a))\mathbf{v}=a\mathbf{v}+(-a)\mathbf{v}$\footnote{La existencia de $-a\in K$ está asegurada ya que $K$ es un campo (ver sección \ref{Def:Campo}).}. Por otro lado, por la propiedad de cerradura del producto de un vector por un escalar $a\mathbf{v}\in V$ y, por la existencia de inversos aditivos, $\exists\hspace{1mm}-(a\mathbf{v})\in V$ tal que $a\mathbf{v}+[-(a\mathbf{v})]=\mathbf{0}$. Por lo tanto, tenemos que $\mathbf{0}=\mathbf{0}\iff a\mathbf{v}+(-a)\mathbf{v}=a\mathbf{v}+[-(a\mathbf{v})]\iff (-a)\mathbf{v}=-(a\mathbf{v})$ por la ley de cancelación. En particular, $(-1)\mathbf{v}=-\mathbf{v}$; por la propiedad de compatibilidad del producto de un vector por un escalar con el producto entre escalares, se sigue que $a(-\mathbf{v})=a[(-1)\mathbf{v}]=[a(-1)]\mathbf{v}=(-a)\mathbf{v}=-(a\mathbf{v})$.
\end{enumerate}
\end{proof}
\end{teorema}



\newpage
\subsection{Interpretación geométrica de las operaciones de los espacios vectoriales}

Como se mencionó en una nota al final de la sección \ref{Ejem:Espacios_vectoriales}, podemos desarrollar nuestra intuición sobre muchos temas del álgebra lineal trabajando en espacios vectoriales \emph{visualizables}, y luego extender esa intuición a espacios vectoriales más generales. Por ende, ahora haremos hincapié en la interpretacción geométrica de las operaciones de los espacios vectoriales, en particular, en los espacios vectoriales reales $\mathbb{R}^2$ y $\mathbb{R}^3$, así como en el espacio vectorial complejo $\mathbb{C}$. Aquí, retomaremos muchas de las ideas presentadas en dicha sección.

\subsubsection{En el espacio vectorial real $\mathbb{R}^2$} \label{Ejem:En_R^2}

Cada vector $\begin{bmatrix}a&b\end{bmatrix}^T \in \mathbb{R}^2$ tiene una correspondencia uno a uno (o \emph{biunívoca}) con una flecha en el plano cartesiano que tiene cola en el origen y punta en la coordenada $(a,b)$ correspondiente.

%\begin{figure}
%    \centering
%    \includegraphics{}
%    \caption{Caption}
%    \label{fig:my_label}
%\end{figure}

Recordemos que tanto los vectores de $\mathbb{R}^2$ como las coordenadas del plano cartesiano representan \emph{pares ordenados} o  \emph{2-tuplas}. Por definición, si $a\neq b$, entonces el par ordenado $(a,b)\neq(b,a)$. En el caso de los vectores, esto se muestra claramente en la siguiente Figura.

%\begin{figure}
%    \centering
%    \includegraphics{}
%    \caption{Caption}
%    \label{fig:my_label}
%\end{figure}{}

En este espacio, la suma vectorial se define como $\begin{bmatrix}a&b\end{bmatrix}^T+\begin{bmatrix}c&d\end{bmatrix}^T\equiv\begin{bmatrix}a+c&b+d\end{bmatrix}^T$. Podemos calcular, por ejemplo, la suma $\begin{bmatrix}2&1\end{bmatrix}^T+\begin{bmatrix}1&3\end{bmatrix}^T=\begin{bmatrix}3&4\end{bmatrix}^T$.

%\begin{figure}
%    \centering
%    \includegraphics{}
%    \caption{Caption}
%    \label{fig:my_label}
%\end{figure}{}

Observemos que, visualmente, esto corresponde a trazar uno de los vectores en el plano cartesiano y luego trazar el otro colocando la cola en la punta del vector anterior, como si ése fuese su origen. Nótese que no importa cuál vector trazamos primero y cuál después, lo cual concuerda con la conmutatividad de la suma vectorial (esta misma analogía visual es válida para la suma de tres o más vectores de $\mathbb{R}^2$). En particular, $\forall\hspace{0.5mm} \mathbf{v} \in \mathbb{R}^2$, $\mathbf{0}+\mathbf{v}=\mathbf{v}$, lo cual concuerda con el hecho de que el vector $\mathbf{v}$ corresponda a una flecha con cola en el origen.

%\begin{figure}
%    \centering
%    \includegraphics{}
%    \caption{Caption}
%    \label{fig:my_label}
%\end{figure}{}

Además, en este espacio, el producto de un vector por un escalar se define como $c\begin{bmatrix}a&b\end{bmatrix}^T\equiv\begin{bmatrix}ca&cb\end{bmatrix}^T$. Podemos calcular, por ejemplo, los productos $(\frac{1}{2})\begin{bmatrix}2&1\end{bmatrix}^T=\begin{bmatrix}1&0.5\end{bmatrix}^T$ y $(-1.2)\begin{bmatrix}1&3\end{bmatrix}^T=\begin{bmatrix}-1.2&-3.6\end{bmatrix}^T$. La representación gráfica de estas operaciones se muestra en la siguiente figura.

%\begin{figure}
%    \centering
%    \includegraphics{}
%    \caption{Caption}
%    \label{fig:my_label}
%\end{figure}{}

Como podemos observar, el primer producto redujo la longitud del vector sin cambiar su sentido, mientras que el segundo producto aumentó la longitud del vector, a la vez que invirtió su sentido; sin embargo, en ambos casos, el producto de un vector por un escalar no cambió la \emph{dirección} de los vectores\textemdash es decir, los mantuvo en la misma \emph{línea}. En general, si el escalar $c\in\mathbb{R}$ que multiplica al vector tiene $|c|>1$, lo \emph{alarga}; si tiene $|c|<1$, lo acorta; finalmente, si tiene $|c|=1$, no cambia su longitud. Por este cambio de longitud es que al producto de un vector por un escalar también se le conoce como \emph{reescalamiento}. Además, si $c>0$, el vector mantiene su misma dirección y sentido (sigue en la misma línea y apunta hacia el mismo lado) mientras que, si $c<0$, el vector conserva su dirección pero se invierte su sentido (sigue en la misma línea pero apunta hacia el lado opuesto); si $c=0$ entonces el vector automáticamente se convierte en el vector nulo $\begin{bmatrix}0&0\end{bmatrix}^T$, como se demostró algebráicamente en el Teorema 2.1.

Así, en general, si combinamos las operaciones de suma vectorial y producto de un vector por un escalar, visualmente lo que estaremos haciendo será \emph{combinar líneas} con diferentes longitudes, direcciones y sentidos en el plano cartesiano.

\vspace{3mm}

Nota: El vector nulo $\mathbf{0}=\begin{bmatrix}0&0\end{bmatrix}^T$ (también llamado \emph{vector origen}) no tiene longitud, ya que es el único donde la cola y la punta de su flecha coinciden. Además, se dice que tampoco tiene dirección ni sentido\footnote{Alternativamente, se dice que tiene \emph{todas las direcciones} y \emph{todos los sentidos simultáneamente}: en la práctica, ambas interpretaciones son equivalentes, pero la primera puede ser más fácil de asimilar.}. Si asumimos que este vector no tiene longitud, dirección ni sentido, entonces queda claro por qué cualquier reescalamiento de este vector no lo modifica, como se demostró en el Teorema 2.2.

\subsubsection{En el espacio vectorial real $\mathbb{R}^3$}

La suma vectorial y el producto de un vector por un escalar (o \emph{reescalamiento}) en el espacio vectorial real $\mathbb{R}^3$ tienen la misma interpretación geométrica que en $\mathbb{R}^2$, con una dimensión extra añadida. Esto es de esperarse, ya que las definiciones de estas operaciones y las correspondencias entre vectores y flechas que salen del origen a una coordenada específica son análogas en ambos espacios vectoriales.

\subsubsection{En el espacio vectorial complejo $\mathbb{C}$}

Como hemos visto, el plano cartesiano nos sirve para representar vectores con dos entradas reales. De manera similar, el \emph{plano complejo} \textemdash con un eje de números \emph{reales} (por convención, el horizontal) y otro eje perpendicular a él con números \emph{imaginarios}\footnote{Los números imaginarios son aquellos números complejos con la parte real igual a cero, i.e. $0+ib=ib\in\mathbb{C}$, donde $b$ es un número real. En otras palabras, son el resultado de multiplicar el número imaginario $i$ por cualquier número real.}\textemdash\hspace{0.5mm} nos sirve para representar vectores con una entrada compleja. Así, cada vector de una entrada compleja $\begin{bmatrix}a+ib\end{bmatrix}$ tiene una correspondencia uno a uno con una flecha con cola en el origen del plano y flecha en la coordenada $(a,ib)$ del plano complejo, la cual corresponde a, desde el origen, moverse $a$ unidades sobre el eje real y $b$ unidades sobre el eje imaginario.

De la siguiente definición de suma vectorial $\begin{bmatrix}a+ib\end{bmatrix}+\begin{bmatrix}c+id\end{bmatrix}\equiv\begin{bmatrix}(a+c)+(b+d)i\end{bmatrix}$ se sigue que la suma vectorial entre vectores de $\mathbb{C}$ tenga la misma interpretación geométrica que aquella entre vectores de $\mathbb{R}^2$. Por ejemplo, si calculamos $\begin{bmatrix}1+2i\end{bmatrix}+\begin{bmatrix}3+2i\end{bmatrix}=\begin{bmatrix}4+4i\end{bmatrix}$, podemos representarlo visulamente en la siguiente Figura.

%\begin{figure}
%    \centering
%    \includegraphics{}
%    \caption{Caption}
%    \label{fig:my_label}
%\end{figure}{}

Por definición, el producto de un vector por un escalar es $(q+ir)\begin{bmatrix}s+it\end{bmatrix}\equiv\begin{bmatrix}(qs-rt)+(qt+rs)i\end{bmatrix}$. Notemos que, en particular, si la parte imaginaria del escalar es nula (i.e., si $r=0$), entonces el escalar es un número real y el producto resultante es $(q)\begin{bmatrix}s+it\end{bmatrix}\equiv\begin{bmatrix}(qs)+(qt)i\end{bmatrix}$, por lo cual geométricamente sólo se produce un reescalamiento completamente análogo al discutido en la sección \ref{Ejem:En_R^2}.

%\begin{figure}
%    \centering
%    \includegraphics{}
%    \caption{Caption}
%    \label{fig:my_label}
%\end{figure}{}

En cambio, observemos qué sucede si la parte real del escalar es nula y la parte imaginaria es igual a $1$ (i.e., si $q=0$ y $r=1$). Tomemos, por ejemplo, el vector $\begin{bmatrix}2+2i\end{bmatrix}$. Al hacer el producto de este vector por el escalar $i$ obtenemos $\begin{bmatrix}-2+2i\end{bmatrix}=\begin{bmatrix}2+2i\end{bmatrix}$. Si, en cambio, hacemos el producto de este mismo vector por el escalar $-i$, obtenemos como resultado $(-i)\begin{bmatrix}2+2i\end{bmatrix}=\begin{bmatrix}2-2i\end{bmatrix}$. Ambas operaciones se muestran de manera visual en la siguiente Figura.

%\begin{figure}
%    \centering
%    \includegraphics{}
%    \caption{Caption}
%    \label{fig:my_label}
%\end{figure}{}

Aquí vemos que hacer el producto de un vector por el escalar $i$ \emph{equivale a hacer una rotación de $90^\circ$ ó $\frac{\pi}{2}$ radianes}. Análogamente, el producto de un vector por el escalar $-i$ equivale a hacer una rotación de $-90^\circ$ ó $-\frac{\pi}{2}$ radianes. Esto tiene sentido ya que, por ejemplo, multiplicar un vector por el escalar $i$ dos veces \textemdash lo cual geométricamente lo rotaría $\frac{\pi}{2}$ radianes dos veces\textemdash\hspace{0.5mm} es equivalente a multiplicar ese mismo vector por el escalar $i^2=-1$ \textemdash lo cual geométricamente invertiría su sentido. Otro ejemplo es que multiplicarlo por el escalar $i$ cuatro veces \textemdash o rotarlo un total de $4(\frac{\pi}{2})=2\pi$ radiantes\textemdash\hspace{0.5mm} sería equivalente a multiplicarlo por el escalar $(i)^4=1$.

%\begin{figure}
%    \centering
%    \includegraphics{}
%    \caption{Caption}
%    \label{fig:my_label}
%\end{figure}{}

En general, hacer el producto de un vector complejo por un número imaginario positivo lo rotará en el sentido contrario a las manecillas del reloj, mientras que el producto de un vector complejo por un número imaginario negativo lo rotará en el sentido de las manecillas.

Dicho lo anterior, estamos listos para el caso más general: multiplicar un vector complejo $\begin{bmatrix}s+it\end{bmatrix}$ por un escalar complejo $q+ir$ con $q,r\neq0$ \emph{reescalará} el vector en el plano complejo y lo \emph{rotará} en el sentido correspondiente al signo de $r$. Es decir que, en los espacios vectoriales complejos, los escalares no sólamente pueden \emph{reescalar} vectores, sino que también los pueden \emph{rotar}\footnote{El asunto de las \emph{magnitudes específicas} de estos reescalamientos y rotaciones será precisado más adelante.}.

%\begin{figure}
%    \centering
%    \includegraphics{}
%    \caption{Caption}
%    \label{fig:my_label}
%\end{figure}{}

Nuevamente, si combinamos las operaciones de suma vectorial y producto de un vector por un escalar en un espacio vectorial complejo, esta operación se interpreta geométricamente como una combinación de líneas reescaladas y rotadas, cada una con su propia longitud, dirección y sentido.


\newpage
\subsection{Subespacios vectoriales}

En ciertas formas \emph{específicas}, las cuales iremos detallando, los subespacios vectoriales son a los espacios vectoriales lo que los subconjuntos a los conjuntos. Por ejemplo: así como cualquier subconjunto es, en sí mismo, un conjunto, cualquier subespacio vectorial es, en sí mismo, un espacio vectorial.

\subsubsection{Definición de subespacio vectorial} \label{Def:Subespacio_vectorial}

\begin{tcolorbox}
\underline{Def.} Sea un conjunto $V$ sobre un campo $K$ un espacio vectorial. Un \textit{subespacio vectorial} de $V$ es un subconjunto $W\subset V$ sobre el campo $K$ con las operaciones de suma vectorial y producto de un vector por un escalar que cumple las propiedades siguientes:

\begin{center}
\begin{tabular}{lr}
    $\forall\hspace{1.5mm} \mathbf{w},\mathbf{x}\in W \hspace{3mm}\exists \hspace{1.5mm} \mathbf{w}+\mathbf{x}\in W$ & Cerradura de la adición \\ \\ \multirow{2}{0.4\textwidth}{$\forall\hspace{1.5mm} \mathbf{w}\in W, a\in K \hspace{3mm}\exists \hspace{1.5mm} a\mathbf{w}\in W$} & \multirow{2}{0.28\textwidth}{Cerradura del producto de un vector por un escalar} \\ \\ \\
    $\exists \hspace{1.5mm} \mathbf{0}\in W$ t.q. $\mathbf{w}+\mathbf{0}=\mathbf{w}\hspace{3mm}\forall\hspace{1.5mm} \mathbf{w} \in W$ & Elemento identidad de la adición (neutro aditivo). \\ \\
\end{tabular}
\end{center}

\hspace{2.5mm} En este caso, a $W$ se le conoce como un \textit{subespacio vectorial} de $V$ (nuevamente, simplificando la notación); sin embargo, $W$ también es, en sí mismo, un espacio vectorial.

\end{tcolorbox}{}

Observemos que:

\begin{itemize}
    \item Ya que cualquier subespacio vectorial es un espacio vectorial, entonces cualquier subespacio vectorial puede tener subespacios vectoriales subsecuentes\footnote{Esto es más o menos similar al hecho de que cualquier subconjunto puede o no tener subconjuntos subsecuentes.}.
    \item La definición de subespacio vectorial sólo incluye tres propiedades. Esto nos indica que, ya que $W$ es subconjunto de $V$ y ambos espacios vectoriales están definidos sobre el mismo campo $K$, si $W$ cumple explícitamente las tres propiedades mencionadas, las demás propiedades de un espacio vectorial se siguen trivialmente.
    \item Para todo espacio vectorial $V$, $V$ y $\{\mathbf{0}\}$ son subespacios vectoriales de $V$\footnote{Aquí se sobreentiende que los conjuntos $V$ y $\{\mathbf{0}\}$ están definidos como espacios vectoriales sobre el mismo campo $K$.}.
\end{itemize}{}

\subsubsection{Ejemplos de subespacios vectoriales}

El conjunto de todos los pares ordenados $\{\begin{bmatrix} x_1&x_2\end{bmatrix}^T\mathop |\mathop x_1,x_2\in\mathbb{R}\mathop\land\mathop x_1=x_2\}$ es un subespacio vectorial del espacio vectorial real $\mathbb{R}^2$ (o $\mathbb{R}\times \mathbb{R}$).

\vspace{3mm}

El conjunto $\mathbb{R}$ sobre el campo $\mathbb{R}$ es un subespacio vectorial del conjunto $\mathbb{C}$ sobre el mismo campo $\mathbb{R}$.

\vspace{3mm}

Sean $j,k\in\mathbb{N}$ t.q. $j<k$. El conjunto de polinomios de grado $j$ es un subespacio vectorial\footnote{De aquí en adelante, asumiremos que cualquier espacio vectorial $V$ está definido por un conjunto vectorial $V$ sobre el campo $\mathbb{R}$ (espacio vectorial real), a menos que se indique lo contrario.} del espacio vectorial formado por el conjunto de polinomios de grado $k$.

\vspace{3mm}

El conjunto de todas las funciones reales de clase $C^{\infty}$ es un subespacio vectorial del espacio vectorial formado por el conjunto de todas las funciones reales de clase $C^n$ (con $n\in\mathbb{N}$).

\vspace{3mm}

\subsubsection{Algunos teoremas de subespacios vectoriales} \label{Teo:Subespacios_vectoriales}

\begin{teorema} 3 Cualquier intersección de dos subespacios vectoriales de $V$ es un subespacio vectorial de $V$.

\begin{proof}
Sea $V$ sobre $K$ un espacio vectorial y sea $C$ una colección de subespacios vectoriales de $V$ (definidos sobre el mismo campo $K$). Sea $W$ la intersección de los conjuntos vectoriales de $C$, entonces $W$ sobre $K$ es un espacio vectorial por el Teorema 3. Por lo tanto, $W$ trivialmente cumple con la cerradura de la adición y del producto de un vector por un escalar, y contiene al elemento identidad de la adición. En conclusión, $W$ es un subespacio de $V$.
\end{proof}
\end{teorema}

\begin{teorema} 4 Sea $Z$ un subespacio vectorial de $W$ y sea $W$, a su vez, subespacio vectorial de $V$. Entonces $Z$ es un subespacio vectorial de $V$.
\end{teorema}

\noindent La demostración del Teorema 5 se deja como ejercicio. Este último teorema nos muestra otra analogía válida entre subconjuntos y subespacios vectoriales, ya que si $A \subset B$ y $B\subset C \implies A\subset C$.

\subsubsection{Suma y suma directa de espacios vectoriales}

\begin{tcolorbox}
\underline{Def.} Sean $S_1$ y $S_2$ subespacios de un espacio vectorial $V$. Definimos a la \emph{suma de los subespacios vectoriales} $S_1$ y $S_2$ como el espacio vectorial definido por $S_1+S_2=\{\mathbf{x}+\mathbf{y}\mathop|\mathop \mathbf{x}\in S_1, \mathbf{y}\in S_2\}$\footnote{Aquí se sobreentiende que $S_1, S_2$ y $V$ están definidos sobre un mismo campo.}.

\vspace{3mm}

\underline{Def.} Si, además, se cumple que $S_1+S_2=V$ y $S_1 \cap S_2 = \{\mathbf{0}\}$, decimos que el espacio vectorial $V$ es la \emph{suma directa} de $S_1$ y $S_2$, lo cual denotamos como $S_1\oplus S_2=V$.
\end{tcolorbox}

La operación de suma entre subespacios vectoriales en realidad es una suma entre sus \emph{conjuntos vectoriales} \textemdash así como la intersección de dos espacios vectoriales es en realidad una intersección de los conjuntos vectoriales\textemdash; el conjunto resultante de la suma forma un espacio vectorial sobre el mismo campo que define a los subespacios. Observemos que la definición de suma vectorial pide que $S_1$ y $S_2$ sean \emph{subespacios} de un espacio vectorial $V$, y no sólo espacios vectoriales arbitrarios; esto asegura que su suma $S_1+S_2$ también sea un espacio vectorial.

Por otro lado, la suma directa $V_1\oplus V_2\oplus...\oplus V_n=W$ nos da la sensación de que, en cierto sentido, el espacio vectorial $W$ se puede \emph{descomponer} en sus subespacios $V_1, V_2,...,V_n$, dado que el único elemento común entre cualesquiera de estos dos subespacios es el neutro aditivo.

\vspace{3mm}

Para dar un ejemplo: sea $C$ el espacio vectorial de todas las funciones constantes $f(x) = c$ con $c\in\mathbb{R}$, y sea $D$ el de todas las funciones de la forma $f(x) = d x$ para algún $d\in\mathbb{R}$. Sea $P^n$ el espacio vectorial de todos los polinomios de grado $n$, es decir, de todas las funciones con regla de correspondencia $f(x) = c_0 x^1 + c_1 x^1 + ... + c_n x^n$ con $c_i\in\mathbb{R}$, entonces $C\oplus D = P^1$ (de hecho, nótese que $P^0=C$ por lo cual pudimos haber escrito $P^0\oplus D=P^1$ de manera equivalente).

\vspace{3mm}

Volveremos a esta idea de \emph{descomponer un espacio vectorial como una suma directa de sus subespacios vectoriales} cuando veamos los temas de dependencia e independencia lineal, bases y dimensión (tentativamente, en la tercera semana del curso). Antes de eso, veremos otro tipo de operación de los espacios vectoriales, la cual se realiza entre dos vectores y da como resultado un escalar.

\newpage
\subsection{Ejercicios de repaso}

\subsubsection{Campos}

\begin{enumerate}
    \item Demuestra que el conjunto $\mathbb{C}$ junto con las operaciones de suma y multiplicación definidas en la sección \ref{Ejem:Campo_complejo} forma un campo.
\end{enumerate}{}

\subsubsection{Espacios vectoriales} \label{Ejer:Espacios_vectoriales}

\begin{enumerate}
    \item En la definición de espacio vectorial de la sección \ref{Def:Espacio_vectorial} no se especifica que los resultados de las operaciones $\mathbf{u}+\mathbf{v}$ (en la propiedad de \textit{cerradura de la adición}) ni $a\mathbf{v}$ (en \textit{cerradura del producto de un vector por un escalar}) sean únicos. Tampoco se especifica que el elemento identidad de la adición ($\mathbf{0}$), el elemento identidad del producto de un vector por un escalar ($1$), ni los elementos inversos de la adición de cada vector $\mathbf{v}$ ($-\mathbf{v}$) sean únicos. Demuestra que todos los elementos mencionados anteriormente son únicos. 
    \item Sea $F$ un campo arbitrario. ¿Siempre puede definirse un espacio vectorial de $F$ (como conjunto vectorial) sobre sí mismo (como campo)? Si sí, demuéstralo. Si no, da un contraejemplo (nota: repasa los axiomas de campo y ten cuidado con la notación que decidas emplear).
    \item Explica por qué no puede existir un espacio vectorial con vectores reales sobre un campo complejo, pero sí puede haber un espacio vectorial con vectores complejos sobre un campo real.
\end{enumerate}{}

\subsubsection{Interpretación geométrica de las operaciones de los espacios vectoriales}

\begin{enumerate}
    \item 
\end{enumerate}{}

\subsubsection{Subespacios vectoriales}

\begin{enumerate}
    \item En la página 17 del libro \textit{Linear Algebra} de Friedberg, la definición de un subespacio vectorial incluye una cuarta propiedad que deben cumplir (la de existencia de inversos aditivos), la cual no incluí en las notas del curso, ya que es redundante. Explica por qué se puede desechar esta cuarta propiedad sin afectar la definición de subespacio vectorial (nota: ver, por ejemplo, la definición de subespacios vectoriales dada en la página 91 de \textit{Introduction to Linear Algebra} de Lang, que no incluye dicha propiedad).
    \item Demuestra que el conjunto $\{(a,2a,3a)\in\mathbb{R}^3\mathop|\mathop a\in\mathbb{R}\}$ sobre el campo $\mathbb{R}$ es un subespacio vectorial de $\mathbb{R}^3$. ¿Qué representa este conjunto visualmente en $\mathbb{R}^3$?
    \item Demuestra que el espacio vectorial de todas las funciones de una variable real que son derivables y tienen derivada continua (de clase $C^1$) sobre $\mathbb{R}$ es un subespacio vectorial del espacio vectorial de todas las funciones reales continuas (de clase $C^0$) de una variable sobre $\mathbb{R}$. ¿Cómo generalizarías este resultado a que las de clase $C^i$ formen un subespacio de las de clase $C^n$ con $i>n$ sobre el mismo campo $\mathbb{R}$? Argumenta.
    \item Demuestra que el espacio vectorial de todas las funciones polinomiales de una variable real de grado $n$ es un subespacio vectorial del espacio vectorial de todas las funciones reales de una variable de clase $C^m$ para toda $n \geq m$.
    \item Demuestra que $\mathbb{C}^n$ sobre $\mathbb{R}$ es un espacio vectorial; después, demuestra que este mismo espacio vectorial es un subespacio vectorial de $\mathbb{C}^n$ sobre $\mathbb{C}$. Con esto, habrás mostrado que un espacio vectorial real puede ser subespacio de un espacio vectorial complejo, ¿crees que este resultado sea general, o que sólo se cumpla para un cierto tipo de conjuntos vectoriales? Argumenta.
    \item Demuestra el Teorema 5 (ver sección \ref{Teo:Subespacios_vectoriales}).
\end{enumerate}{}







\newpage
\section{Producto escalar (punto), norma, proyecciones y ortogonalidad, producto vectorial (cruz)* y triple producto escalar*}

\vspace{1cm}

\begin{tcolorbox} \label{1:Notación}
\underline{Notación:}

\centering
\begin{tabular}{cc}
    \\
    $(\mathbf{u},\mathbf{v}) \equiv \langle\mathbf{u},\mathbf{v}\rangle \equiv \mathbf{u}\cdot\mathbf{v}$ & producto escalar (punto) entre los vectores $\mathbf{u}$ y $\mathbf{v}$ \\ \\
    $\overline{a} \equiv a^*$ & complejo conjugado de $a$ \\ \\
    $u_i v_i \equiv \sum_{i=1}^n u_i v_i$ & \textit{notación de Einstein} para el producto escalar de dos $n$-tuplas. \\ \\
    $||\mathbf{u}||$ & norma del vector $\mathbf{u}$ \\ \\
    $P_{\mathbf{u}}(\mathbf{v})$ & proyección del vector $\mathbf{v}$ sobre el vector $\mathbf{u}$ \\ \\
    $\mathbf{u}\perp\mathbf{v}$ & ortogonalidad de los vectores $\mathbf{u}$ y $\mathbf{v}$ \\ \\
    $\mathbf{a}\times\mathbf{b}$ & producto vectorial (cruz) de dos vectores $\mathbf{a},\mathbf{b}\in\mathbb{R}^3$ \\ \\
    $\mathbf{a}\cdot\mathbf{b}\times\mathbf{c}$ & triple producto escalar entre tres vectores $\mathbf{a},\mathbf{b},\mathbf{c}\in\mathbb{R}^3$ \\
\end{tabular}
\end{tcolorbox}


\newpage
\subsection{Producto escalar (punto)}

\subsubsection{Definición de producto escalar (punto)} \label{Def:Producto_escalar}

\begin{tcolorbox}
\underline{Def.} Sea $V$ sobre $K$ un espacio vectorial, con $\mathbf{u},\mathbf{v},\mathbf{w}\in V$ y $a\in K$. El \textit{producto escalar} en $V$ $(\cdot\mathop ,\cdot)$ le asocia a cualquier par ordenado de vectores en $V$ un escalar en $K$, y cumple las tres propiedades siguientes:

\begin{tabular}{l}
    \\
    $(\mathbf{u}+\mathbf{w},\mathbf{v}) = (\mathbf{u},\mathbf{v})+(\mathbf{w},\mathbf{v}),$ \\ \\ $(a\mathbf{u},\mathbf{v}) = a(\mathbf{u},\mathbf{v}),$ \\ \\
    $(\mathbf{u},\mathbf{v})=\overline{(\mathbf{v},\mathbf{u})},$ \\ \\
\end{tabular}

donde la barra $\overline{(\mathbf{v},\mathbf{u})}$ denota el complejo conjugado de $(\mathbf{v},\mathbf{u})$\footnote{Recordemos que, en general, nuestro campo $K$ puede ser complejo; en ese caso, el producto escalar en $V$ le asignará a cada par ordenado de dos vectores de $V$ un escalar complejo $c\in\mathbb{C}$.}. Si además, se cumple la propiedad

\begin{tabular}{l}
    \\
    $\forall\hspace{1.5mm}\mathbf{v}\in V, \hspace{1.5mm} (\mathbf{v},\mathbf{v})\geq0; \hspace{1.5mm} (\mathbf{v},\mathbf{v}) = 0 \iff \mathbf{v}=\mathbf{0}$, \\ \\
\end{tabular}

se dice que el producto escalar es \textit{positivo definido}.

\vspace{3mm}
\hspace{2.5mm} Para vectores que son $n-$tuplas (como aquellos de $\mathbb{R}^n$ o $\mathbb{C}^n$), es común que el producto escalar de $\mathbf{u}$ y $\mathbf{v}$ se denote por $\mathbf{u}\cdot\mathbf{v}$, por lo cual también se le conoce como \textit{producto punto}; sin embargo, también existen notaciones alternativas como $\langle \mathbf{u}, \mathbf{v}\rangle$\footnote{Esta notación para el producto escalar es similar a la utilizada en Mecánica Cuántica $\braket{\phi|\psi}$ \textemdash en donde el campo siempre es complejo\textemdash, y esconde un profundo significado, el cual veremos más adelante.}.

\end{tcolorbox}{}

A partir de la definición, observemos lo siguiente:

\begin{itemize}
    \item Las primeras dos propiedades juntas nos dicen que $(\mathbf{u}+a\mathbf{w},\mathbf{v}) = (\mathbf{u},\mathbf{v})+a(\mathbf{w},\mathbf{v})$. Para referirnos a esta propriedad específica en lenguaje matemático, decimos que el producto escalar \textit{es una operación lineal\footnote{Como debes sospechar, el concepto de operación \emph{lineal} es fundamental para el álgebra \emph{lineal}, y lo veremos con más detalle en las secciones de transformaciones y operadores lineales.} en el primer elemento}.
    \item La tercera propiedad nos dice que, si el campo es real, el producto escalar es una operación conmutativa, es decir, que $(\mathbf{u},\mathbf{v})=(\mathbf{v},\mathbf{u})$.
    \item Esta misma propiedad nos dice que, si el campo es complejo, el producto escalar es no conmutativo.
    \item Las primeras tres propiedades juntas nos dicen que, si el espacio vectorial está definido sobre un campo complejo, entonces $(\mathbf{u},\mathbf{v}+a\mathbf{w}) = (\mathbf{u},\mathbf{v})+\overline{a}(\mathbf{u},\mathbf{w})$, donde $\overline{a}$ (ó $a^*$) denota el complejo conjugado del escalar complejo $a$. Esto nos dice que, si el campo es complejo, el producto escalar \textit{es una operación antilineal\footnote{Este concepto también se verá con mayor detenimiento más adelante.} en el segundo elemento}; en cambio, si el campo es real, entonces el producto escalar es lineal en el primer y segundo elemento.
    \item Para un producto escalar positivo definido, el único vector que puede tener como resultado el escalar $0$ al hacer producto escalar consigo mismo es el vector $\mathbf{0}$ (el neutro aditivo que vimos en las secciones \ref{Def:Espacio_vectorial} y \ref{Def:Subespacio_vectorial}).
\end{itemize}{}

\subsubsection{Ejemplos de producto escalar en espacios vectoriales} \label{Ejem:Producto_escalar}

En $\mathbb{R}^2$ el producto escalar se define como $\begin{bmatrix}u_1\\u_2\end{bmatrix}\cdot\begin{bmatrix}v_1\\v_2\end{bmatrix} \equiv u_1v_1+u_2v_2$, donde hemos utilizado la notación de punto, ya que los vectores son 2-tuplas. En general, en el espacio vectorial real $\mathbb{R}^n$ el producto escalar se define como

$$\mathbf{u}\cdot\mathbf{v} = \begin{bmatrix}u_1&u_2&...&u_n\end{bmatrix}^T\cdot\begin{bmatrix}v_1&v_2&...&v_n\end{bmatrix}^T \equiv u_1v_1+u_2v_2+...+u_nv_n=\sum_{i=1}^n u_i v_i\footnote{Para simplificar la notación, en varias áreas de la física se elimina el signo de suma ($\Sigma$) cuando aparecen índices repetidos, siguiendo la convención de que éstos implícitamente indican una suma sobre el índice. Así, $\sum_{i=1}^n u_i v_i$ se puede escribir simplemente como $u_iv_i$. A esto se le conoce como \textit{notación de Einstein} o \textit{convención de suma de Einstein}.}.$$

Para cumplir \emph{todas} las propiedades descritas en la sección \ref{Def:Producto_escalar}, el producto escalar en el espacio vectorial complejo $\mathbb{C}^n$ se debe definir de una forma ligeramente distinta. Sean $\mathbf{a},\mathbf{b}\in\mathbb{C}^n$ vectores con $n$ entradas complejas, entonces el producto escalar se define como

$$(\mathbf{a},\mathbf{b})\equiv\mathbf{a}\cdot\mathbf{b}\equiv \sum_{i=1}^n a_i b_i^*,$$

\noindent es decir, se realiza un producto entre la $i$-ésima entrada de $\mathbf{a}$ y el \emph{complejo conjugado} de la $i$-ésima entrada de $\mathbf{b}$ para cada $i$, y luego se suman estos productos.

\vspace{3mm}

Sea $C^0([a,b])$ el conjunto de todas las funciones de variable real continuas en el intervalo cerrado $[a,b]$ \underline{con integral finita}, entonces podemos definir un producto escalar en el espacio vectorial del conjunto $C^0([a,b])$ sobre el campo $\mathbb{R}$ como

$$(f,g) = \int_{a}^{b} f(x)g(x)dx.$$

\vspace{3mm}

Observemos que, como muestra el último ejemplo, se puede definir un proucto escalar en muchos tipos de espacios vectoriales diferentes, y no sólo en aquellos que tienen como vectores a $n-$tuplas\footnote{Veremos otros tipos de espacios vectoriales donde se pueden definir productos escalares más adelante.}. Nótese además que, en cada caso, el resultado del producto escalar es un escalar del campo sobre el cual está definido el espacio vectorial.

\vspace{3mm}

Para ver más ejemplos de productos escalares pueden revisar \emph{Linear Algebra} de Friedberg (págs. 330-331), \emph{Introduction to Linear Algebra} de Lang (págs. 172-173), \textit{Linear Algebra: A Modern Introduction} de Poole\footnote{Ten en cuenta que el libro de Poole introduce el producto escalar (al cual llama producto punto) únicamente con espacios vectoriales reales, y lo generaliza a espacios vectoriales complejos en secciones posteriores.} (págs. 531-534), etc.


\subsubsection{Propiedades del producto escalar} \label{Prop:Producto_escalar}

Las principales propiedades del producto escalar son:

\begin{center}
    \begin{tabular}{lr}
        $(\mathbf{u},\mathbf{v}) = \overline{(\mathbf{v},\mathbf{u})}$ & Conmutar vectores resulta en la conjugación del escalar \\
        $(\mathbf{u},\mathbf{v}+\mathbf{w}) = (\mathbf{u},\mathbf{v}) + (\mathbf{u},\mathbf{w})$ & Distributividad con respecto a la suma vectorial \\
        \multirow{2}{0.35\textwidth}{$(a\mathbf{u},\mathbf{v}) = a(\mathbf{u},\mathbf{v})$ \\ $(\mathbf{u},a\mathbf{v}) = a^*(\mathbf{u},\mathbf{v})$} & \multirow{2}{0.47\textwidth}{Compatibilidad con el producto de un vector por un escalar.} \\ \\
    \end{tabular}{}
\end{center}{}

En particular, el producto escalar definido \underline{sobre un espacio vectorial real} es una operación lineal en ambas entradas (o \emph{bilineal}), es decir, que $$(a_1\mathbf{u_1}+...+a_n\mathbf{u_n},\mathbf{v})=a_1(\mathbf{u_1},\mathbf{v})+...+a_n(\mathbf{u_n},\mathbf{v})\hspace{1mm};\hspace{3mm} (\mathbf{u},b_1\mathbf{v_1}+...+b_n\mathbf{v_n})=b_1(\mathbf{u},\mathbf{v_1})+...+b_n(\mathbf{u},\mathbf{v_n}).$$

\newpage
\subsection{Norma}

\subsubsection{Definición de norma} \label{Def:Norma}

\begin{tcolorbox}
\underline{Def.} Una \textit{norma} es una operación $||\cdot||$ que toma sólo un vector y devuelve un escalar, y que cumple las siguientes propiedades:

\begin{center}
    \begin{tabular}{lr}
        $||\mathbf{u}+\mathbf{v}|| \leq ||\mathbf{u}|| + ||\mathbf{v}||$ & Satisface la desigualdad del triángulo \\ \\
        $||a\mathbf{u}|| = |a|\hspace{0.5mm}||\mathbf{u}||$ & Es escalable de forma absoluta \\ \\
        $||\mathbf{u}||\geq0; \hspace{1.5mm}||\mathbf{u}||=0\iff \mathbf{u}=\mathbf{0}$ & Es positivo definida.
    \end{tabular}
\end{center}

\end{tcolorbox}{}

\subsubsection{Ejemplos de norma}

Tanto en el espacio vectorial real $\mathbb{R}$ como en el espacio vectorial complejo $\mathbb{C}$ se puede definir una norma como $$||x|| = |x|$$

\noindent para cualquier vector $x$ de dichos espacios, ya que el valor absoluto cumple trivialmente las propiedades de la sección \ref{Def:Norma} (lo cual quizá demostraste en tu curso de Cálculo I). Como recordatorio, las definiciones del valor absoluto son $$|r| = +\sqrt{r^2} \hspace{3mm}\forall\hspace{0.5mm}r\in\mathbb{
R}; \hspace{3mm} |c| = +\sqrt{cc^*}\hspace{3mm}\forall\hspace{0.5mm}c\in\mathbb{C}.$$

\noindent A ésta se le conoce como la \emph{norma del valor absoluto}\footnote{En el caso de los números reales, el valor absoluto equivale a cambiar los signos de los números negativos por signos positivos y no hacerle nada a los número no negativos.}. En el caso real, esta norma se interpreta geométricamente como la distancia entre el origen de la recta real y el punto correspondiente al valor $r$ o, equivalentemente, como la longitud de la flecha que tiene cola en $0$ y punta en $r$; en el caso complejo, se interpreta como la distancia mínima (o \emph{euclideana}) entre el origen del plano complejo y el punto correspondiente al valor complejo $c$ o, equivalentemente, como la longitud de la flecha que tiene cola en el origen del plano complejo y punta en $c$. 

\begin{comment}

\begin{figure}
    \centering
    \includegraphics{}
    \caption{Caption}
    \label{fig:my_label}
\end{figure}

\end{comment}

\vspace{3mm}

Para los vectores que son $n-$tuplas, la norma básica se define como $$||\mathbf{u}|| = +\sqrt{(\mathbf{u},\mathbf{u})} = +\sqrt{\mathbf{u}\cdot\mathbf{u}},$$ \noindent de acuerdo a las definiciones de productos escalares para $n-$tuplas dadas en la sección \ref{Ejem:Producto_escalar}. Esta norma se interpreta geométricamente como la longitud de la flecha que tiene cola en el origen del espacio vectorial y punta en la coordenada dada por las entradas del vector $\mathbf{u}$, la cual es equivalente a la distancia euclideana entre estos dos puntos. Por ende, a esta norma se le conoce como \emph{norma euclideana}\footnote{Observemos que la norma del valor absoluto es simplemente un caso particular de la norma euclideana para los espacios vectoriales $\mathbb{R}$ y $\mathbb{C}$, tanto en su definición algebráica como en su interpretación geométrica .}. En particular, a los espacios vectoriales reales $\mathbb{R}^n$ con esta norma se les conoce como \emph{espacios vectoriales euclídeanos} (o \emph{euclideanos})\footnote{Estos son los espacios básicos que se utilizan en geometría analítica y cálculo diferencial e integral con funciones de una o más variables reales.}. Ya que la norma está asociada geométricamente a la longitud de la flecha que representa un vector, también se dice que está relacionada con la \emph{magnitud} de un vector.

\vspace{3mm}

También podemos definir una norma a partir del producto escalar definido en el último ejemplo de la sección \ref{Ejem:Producto_escalar} como $$||f|| = +\sqrt{(f,f)} = +\sqrt{\int_a^b f(x)f(x) dx}.$$ \noindent Este resultado se puede generalizar a cualquier producto escalar positivo definido (ver la sección \ref{Ejer:Norma}). Entonces, se dice que este tipo de normas son \emph{inducidas} por un producto escalar positivo definido; en general, las normas de este tipo serán las más recurrentes durante este curso, aunque veremos otros tipos de normas que no son inducidas por productos escalares más adelante.


\newpage
\subsection{Proyecciones y ortogonalidad}







\newpage
\subsection{Producto vectorial (cruz)*} 

El producto vectorial es una operación de gran utilidad, principalmente en algunas ramas de la física clásica (en las cuales se trabaja con vectores en $\mathbb{R}^3$), y tiene una interpretación geométrica bastante rica, por lo cual se menciona brevemente en estas notas. En esta sección también se mencionan por primera vez los temas de matrices, determinantes y vectores canónicos, los cuales asumo que ya debes conocer, aunque más adelante en el curso los repasaremos.

\subsubsection{Definición del producto vectorial (cruz)} \label{Def:Producto_vectorial}

En espacios vectoriales donde los vectores son 3-tuplas (como $\mathbb{R}$ o $\mathbb{C}$), es decir, elementos de la forma $\mathbf{u}=(u_1,u_2,u_3)$, se puede definir una operación entre dos vectores que da como resultado un tercer vector, el cual es ortogonal\footnote{El tema de ortogonalidad lo veremos a detalle más adelante en el curso, aquí se menciona por completez.} a los dos anteriores. A esta operación se le conoce como \emph{producto vectorial}.

\begin{tcolorbox}
\underline{Def.} En $\mathbb{R}^3$, el producto cruz entre dos vectores $\mathbf{r}=(r_1,r_2,r_3)$ y $\mathbf{s}=(s_1,s_2,s_3)$ se define como

$$\mathbf{r}\times\mathbf{s} \equiv (r_2s_3-r_3s_2,r_3s_1-r_1s_3,r_1s_2-r_2s_1).$$

\noindent Dado el símbolo ($\times$) utilizado para denotar esta operación, también se le conoce como \emph{producto cruz}.
\end{tcolorbox}{}

Algebráicamente, el producto cruz $\mathbf{r}\times\mathbf{s}$ también puede ser visto como el determinante de una matriz de $3\times3$ como sigue:

$$\mathbf{r}\times\mathbf{s} = \text{det} \begin{vmatrix} \mathbf{i}&\mathbf{j}&\mathbf{k} \\ r_1&r_2&r_3 \\ s_1&s_2&s_3 \\
\end{vmatrix} = \mathbf{i}(r_2s_3-r_3s_2)+\mathbf{j}(r_3s_1-r_1s_3)+\mathbf{k}(r_1s_2-r_2s_1),$$

\noindent donde $\mathbf{i},\mathbf{j}$ y $\mathbf{k}$ son los vectores canónicos de $\mathbb{R}^3$ $(1,0,0)$, $(0,1,0)$ y $(0,0,1)$, respectivamente.

Geométricamente, si nombramos como $\theta$ al mínimo ángulo de separación entre dos vectores $\mathbf{r}$ y $\mathbf{s}\in\mathbb{R}^3$, entonces podemos hacer la definición equivalente $\mathbf{r}\times\mathbf{s}=||\mathbf{r}|| \hspace{0.5mm}  ||\mathbf{s}||\sin\theta$, donde $||\mathbf{v}||$ indica la norma del vector $\mathbf{v}$. En este caso, la magnitud del producto vectorial se interpreta como la magnitud del área del paralelepípedo formado por los vectores $\mathbf{r}$ y $\mathbf{s}$. De aquí se sigue que el producto vectorial de dos vectores ortogonales sea igual a $1$, mientras que el producto cruz de dos vectores colineales sea $0$ (en particular, el producto cruz de cualquier vector consigo mismo es igual a $0$). Además, de ambas definiciones (algebráica y geométrica) se sigue que el producto vectorial de cualquier vector en $\mathbb{R}^3$ con el vector nulo ($\mathbf{0}$) sea $\mathbf{0}$.

\subsubsection{Propiedades del producto vectorial (cruz)} \label{Prop:Producto_vectorial}

Las siguientes propiedades son fáciles de demostrar para cualesquiera $\mathbf{r},\mathbf{s},\mathbf{t}\in\mathbb{R}^3$ a partir de la definición de la sección \ref{Def:Producto_vectorial} (¡inténtalo!):

\begin{center}
\begin{tabular}{lr}
    $\mathbf{s}\times\mathbf{r} = -\mathbf{r}\times\mathbf{s}$ & Anticonmutatividad del producto vectorial \\
    $\mathbf{r}\times(\mathbf{s}\times\mathbf{t})\neq(\mathbf{r}\times\mathbf{s})\times\mathbf{t}$ & No asociatividad del producto vectorial \\
    $\mathbf{r}+(\mathbf{s}\times\mathbf{t}) = \mathbf{r}\times\mathbf{s}+\mathbf{r}\times\mathbf{t}$ & Distributividad bajo la suma vectorial \\
    $(a\mathbf{r}\times\mathbf{s}) = a(\mathbf{r}\times\mathbf{s})$ & Compatibilidad con el producto de un vector por un escalar\\
    $\mathbf{r}\times(\mathbf{s}\times\mathbf{t})+\mathbf{s}\times(\mathbf{t}\times\mathbf{r})+\mathbf{t}\times(\mathbf{r}\times\mathbf{s}) = \mathbf{0}$ & Identidad de Jacobi para el producto vectorial.
\end{tabular}
\end{center}



\newpage
\subsection{Triple producto escalar*}

Al unir el producto escalar con el producto vectorial, se obtiene una operación llamada \emph{triple producto escalar} entre tres vectores, que da como resultado un escalar. La magnitud del triple producto escalar se interpreta geométricamente como la magnitud del volumen del paralelepípedo formado por los tres vectores.

\subsubsection{Definición de triple producto escalar}
\begin{tcolorbox}
\underline{Def.} El \emph{triple producto escalar} entre tres vectores $\mathbf{r},\mathbf{s},\mathbf{t}\in\mathbb{R}^3$ se define como

$$\mathbf{r}\cdot\mathbf{s}\times\mathbf{t} \equiv r_1(s_2t_3-s_3t_2) + r_2(s_3t_1-s_1t_3) + r_3(s_1t_2-s_2t_1),$$

\noindent es decir, primero se realiza el producto vectorial $\mathbf{s}\times\mathbf{t}$ y luego se realiza el producto escalar entre el vector resultante de esa operación y $\mathbf{r}$.
\end{tcolorbox}

Recordemos que el producto escalar se realiza entre dos vectores y da como resultado un escalar, mientras que el producto vectorial se realiza entre dos vectores pero da como resultado un vector. Por lo tanto, no hay ambigüedad en la expresión $\mathbf{r}\cdot\mathbf{s}\times\mathbf{t}$, ya que la única forma lógica de juntar estas dos operaciones es realizando primero el producto vectorial y después el producto escalar; por lo tanto, escribir esta operación como $\mathbf{r}\cdot(\mathbf{s}\times\mathbf{t})$ sería redundante.

Al igual que el producto vectorial, el triple producto escalar también puede ser visto algebráicamente como el determinante de una matriz:

$$\mathbf{r}\cdot\mathbf{s}\times\mathbf{t} = \begin{vmatrix} r_1&r_2&r_3 \\ s_1&s_2&s_3 \\ t_1&t_2&t_3 \end{vmatrix}.$$

Además, recordando que el producto escalar de dos vectores ortogonales es igual a $0$ y que el vector resultante de la operación $\mathbf{r}\times\mathbf{s}$ es ortogonal tanto a $\mathbf{r}$ como a $\mathbf{s}$, se sigue directamente que $\mathbf{r}\cdot\mathbf{r}\times\mathbf{s}=0$.



\subsubsection{Propiedades del triple producto escalar} \label{Prop:Triple_producto_escalar}

\begin{center}
\begin{tabular}{lr}
    $\mathbf{r}\cdot\mathbf{s}\times\mathbf{t} = \mathbf{s}\cdot\mathbf{t}\times\mathbf{r} = \mathbf{t}\cdot\mathbf{r}\times\mathbf{s}$ & Conmutatividad bajo permutaciones cíclicas \\
    $\mathbf{r}\cdot\mathbf{s}\times\mathbf{t} = \mathbf{r}\times\mathbf{s}\cdot\mathbf{t}$ & Invariancia bajo intercambio de operadores \\
    $\mathbf{r}\cdot\mathbf{s}\times\mathbf{t} = -\mathbf{r}\cdot\mathbf{t}\times\mathbf{s} = -\mathbf{s}\cdot\mathbf{r}\times\mathbf{t} = -\mathbf{t}\cdot\mathbf{s}\times\mathbf{r}$ & Anticonmutatividad bajo intercambio de dos vectores \\
\end{tabular}
\end{center}

Nota: también se puede definir un \emph{triple producto vectorial} entre tres vectores que da como resultado un vector como $\mathbf{r}\times\mathbf{s}\times\mathbf{t}$. Éste es de gran utilidad para hacer demostraciones de algunas identidades en cálculo vectorial en $\mathbb{R}^3$.

\newpage
\subsection{Ejercicios de repaso}

\subsubsection{Producto escalar (punto)}

\begin{enumerate}
    \item Explica por qué no se puede definir un producto escalar de la forma $(f,g)=\int_a^b f(x)g(x)dx$ para funciones integrables $f$ y $g$ con integrales con valor infinito en $[a,b]$ (pista: piensa en los sinónimos que conoces de \emph{producto escalar}).
    \item Sea $\mathbb{C}^2$ sobre $\mathbb{C}$ un espacio vectorial con $\mathbf{q},\mathbf{r}\in\mathbb{C}^2$. Explica cuál(es) de las propiedades de la sección \ref{Prop:Producto_escalar} no se cumpliría(n) si definiéramos el producto escalar en este espacio vectorial ingenuamente como $\mathbf{q}\cdot\mathbf{r}=\sum_{i=1}^n q_ir_i$. Nótese que esta misma definición de producto escalar nos generaría problemas al intentar usarla para definir una norma en este espacio.
\end{enumerate}

\subsubsection{Norma} \label{Ejer:Norma}

\begin{itemize}
\item ¿El conjunto $\{(x_1,x_2,...,x_n)\mathop|\mathop x_i\in\mathbb{R} \land ||(x_1,x_2,...,x_n)||\leq1\}$ puede formar un espacio vectorial sobre el campo $\mathbb{R}$? Argumenta.
\item Demuestra la desigualdad del triángulo $||\mathbf{a}+\mathbf{b}|| \leq ||\mathbf{a}||+||\mathbf{b}||$ para cualesquiera dos vectores en un espacio con producto escalar positivo definido (pista: usa la desigualdad de Schwarz). Con esto, habrás demostrado que a partir de cualquier producto escalar positivo definido $(\mathbf{a},\mathbf{b})$ se puede definir una norma como $||\mathbf{a}|| = +\sqrt{(\mathbf{a},\mathbf{a})}$. ¿No te sientes poderos@?
\end{itemize}


\subsubsection{Proyecciones y ortogonalidad}


\subsubsection{Producto vectorial (cruz)*}

\begin{itemize}
    \item Demuestra la identidad de Jacobi para el producto vectorial $\mathbf{r}\times(\mathbf{s}\times\mathbf{t})+\mathbf{s}\times(\mathbf{t}\times\mathbf{r})+\mathbf{t}\times(\mathbf{r}\times\mathbf{s}) = \mathbf{0}$.
    \item Da una definición de un producto cruz entre vectores de $\mathbb{C}^3$ tal que se mantengan las propiedades vistas en la sección \ref{Prop:Producto_vectorial}.
\end{itemize}{}

\subsubsection{Triple producto escalar*}

\begin{itemize}
    \item Demuestra las propiedades del triple producto escalar de la sección \ref{Prop:Triple_producto_escalar}.
    \item Calcula el triple producto escalar $\mathbf{u}\cdot\mathbf{v}\times\mathbf{w}$ con $\mathbf{u} = (1,3,6)$, $\mathbf{v}=(2,-5,13)$ y $\mathbf{w}=(7,21,42)$.
\end{itemize}{}



\vspace{5cm}

\begin{tcolorbox}
\begin{center}
    \textbf{Nota aclaratoria: \emph{Sobre nombres y traducciones...}}
\end{center}

\hspace{2.5mm}Como seguramente habrás notado al leer los libros recomendados en la sección \ref{Bibliografía}, al producto escalar en inglés se le conoce como \emph{inner product}, y a los espacios vectoriales dotados de un producto escalar se les conoce como \emph{inner product spaces}. En español, al producto escalar también se le conoce como \emph{producto interior}; sin embargo, existe otro tipo de producto diferente al producto escalar al cual en inglés, desafortunadamente, le llaman \emph{interior product}.

\hspace{2.5mm}Esto significa que, en inglés, la convención es que \emph{scalar product} e \emph{interior product} sean operaciones diferentes mientras que, en español, la convención es que \emph{producto escalar} y \emph{producto interior} se refieran a la misma operación. Algunos textos en español utilizan \emph{producto interno} (en vez de producto interior) como sinónimo de \emph{producto escalar} para homologar los nombres con los utilizados en inglés pero, por ahora, las convenciones preponderantes en español e inglés no permiten una traducción directa.

\hspace{2.5mm}Por lo anterior, en estas notas decidí usar únicamente el nombre de \emph{producto escalar} para la operación entre dos vectores que da como resultado un escalar (la cual acostumbramos llamar \emph{producto punto} cuando esos vectores son $n$-tuplas), pero es importante que sepan que esta operación es equivalente al \emph{\underline{inner} product} de los textos en inglés.
\end{tcolorbox}




\newpage
\section{Combinaciones lineales, espacio generado y conjunto generador, dependencia e independencia lineal}






\end{document}
